\documentclass{article}

\usepackage{Sweave}
\begin{document}
\Sconcordance{concordance:PracticasResueltas.tex:PracticasResueltas.Rnw:%
1 2 1 1 0 1 1}
\Sconcordance{concordance:PracticasResueltas.tex:./Portada.Rnw:ofs 5:%
1 34 1}
\Sconcordance{concordance:PracticasResueltas.tex:PracticasResueltas.Rnw:ofs 40:%
6 1 1}
\Sconcordance{concordance:PracticasResueltas.tex:./practica2.Rnw:ofs 42:%
1 2 1 1 2 6 0 1 1 5 0 2 1 5 0 1 1 5 0 1 1 5 0 1 1 5 0 2 1 5 0 3 1 5 0 2 %
1 5 0 2 1 5 0 1 1 5 0 1 1 5 0 1 1 5 0 1 1 7 0 1 1 5 0 1 1 5 0 2 1 5 0 1 %
1 6 0 2 1 6 0 1 1 5 0 1 1 5 0 1 1 5 0 1 1 5 0 1 1 5 0 1 1 5 0 2 1 5 0 1 %
1 5 0 3 1 8 0 2 1 7 0 1 1 5 0 1 1 5 0 1 1 6 0 1 1 5 0 1 1 5 0 2 1 8 0 1 %
1 5 0 1 1 5 0 1 1 6 0 1 1 11 0 1 1 12 0 1 1 8 0 1 1 6 0 1 1 7 0 1 1 7 0 %
1 1 5 0 1 1 8 0 1 1 8 0 1 1 7 0 1 1 5 0 2 1 11 0 1 1 5 0 1 2 9 0 1 1 7 %
0 1 1 8 0 2 1 8 0 2 1 8 0 2 1 5 0 2 1 5 0 2 1 8 0 2 1 35 0 1 13}
\Sconcordance{concordance:PracticasResueltas.tex:PracticasResueltas.Rnw:ofs 524:%
9}
\Sconcordance{concordance:PracticasResueltas.tex:./practica3.Rnw:ofs 525:%
1 1 1 1 2 1 0 1 1 5 0 2 1 5 0 2 1 6 0 2 1 6 0 1 1 5 0 1 1 5 0 1 1 5 0 1 %
1 5 0 1 1 5 0 1 1 6 0 2 1 15 0 1 1 5 0 1 1 5 0 1 1 6 0 1 1 5 0 1 1 6 0 %
1 1 5 0 1 1 5 0 1 1 5 0 1 1 5 0 1 1 6 0 1 1 6 0 1 1 18 0 2 1 7 0 1 1 11 %
0 1 1 5 0 1 3 10 0 3 1 6 0 1 2 1 1 14 0 2 1 8 0 2 1 25 0 2 1 25 0 2 1 %
25 0 1 1 5 0 1 2 10 0 1 1 5 0 2 1 9 0 1 1 7 0 2 1 8 0 1 1 5 0 2 1 9 0 2 %
1 11 0 1 5}
\Sconcordance{concordance:PracticasResueltas.tex:PracticasResueltas.Rnw:ofs 938:%
11}
\Sconcordance{concordance:PracticasResueltas.tex:./practica4.Rnw:ofs 939:%
1 20 1}
\Sconcordance{concordance:PracticasResueltas.tex:PracticasResueltas.Rnw:ofs 960:%
13}
\Sconcordance{concordance:PracticasResueltas.tex:./practica5.Rnw:ofs 961:%
1 1 1 1 2 1 0 1 1 5 0 1 1 5 0 1 1 5 0 2 1 5 0 2 1 5 0 1 8 6 0 1 4 3 0 2 %
1 5 0 3 1 1 7 6 0 2 1 5 0 1 2 1 1 5 0 1 8 5 0 2 1 1 6 3 0 2 1 5 0 1 8 %
10 0 1 4 1 0 3 1 1 2 6 0 1 5 1 1}
\Sconcordance{concordance:PracticasResueltas.tex:PracticasResueltas.Rnw:ofs 1083:%
15}
\Sconcordance{concordance:PracticasResueltas.tex:./practica6.Rnw:ofs 1084:%
1 1 1 1 5 9 0 1 1 6 0 2 1 1 2 7 0 1 1 7 0 1 2 6 0 1 3 1 0 1 3 1 0 1 3 1 %
0 1 3 1 2 1 0 1 3 1 1 8 0 1 1 5 0 1 1 3 0 1 2 2 1}
\Sconcordance{concordance:PracticasResueltas.tex:PracticasResueltas.Rnw:ofs 1160:%
17}
\Sconcordance{concordance:PracticasResueltas.tex:./practica7.Rnw:ofs 1161:%
1 1 1 1 2 1 0 2 1 5 0 1 1 5 0 1 2 1 1 8 0 1 1 5 0 1 2 1 1 5 0 1 3 1 0 1 %
2 7 0 1 1 9 0 1 1 6 0 1 1 8 0 1 2 3 1 11 0 2 1 11 0 1 2 5 0 1 2 1 1 5 0 %
1 2 5 0 1 2 5 0 1 2 5 0 1 1 5 0 1 2 6 0 1 2 6 0 1 2 6 0 1 2 5 0 1 4 1 0 %
1 4 1 0 1 3 1 2 1 0 2 2 6 0 1 5 2 1}
\Sconcordance{concordance:PracticasResueltas.tex:PracticasResueltas.Rnw:ofs 1347:%
19}
\Sconcordance{concordance:PracticasResueltas.tex:./pr�ctica8.Rnw:ofs 1348:%
1 152 1}
\Sconcordance{concordance:PracticasResueltas.tex:PracticasResueltas.Rnw:ofs 1501:%
21}
\Sconcordance{concordance:PracticasResueltas.tex:./practica9.Rnw:ofs 1502:%
1 64 1}
\Sconcordance{concordance:PracticasResueltas.tex:PracticasResueltas.Rnw:ofs 1567:%
23}
\Sconcordance{concordance:PracticasResueltas.tex:./practica10.Rnw:ofs 1568:%
1 1 1 1 2 6 0 1 1 5 0 1 1 5 0 1 1 10 0 1 2 1 0 1 2 1 0 1 1 10 0 1 1 10 %
0 2 1 9 0 1 2 1 0 1 2 12 0 1 5 3 0 1 5 3 0 1 3 1 0 1 1 8 0 1 2 20 0 1 1 %
5 0 1 4 10 0 1 4 7 0 1 1 5 0 1 1 15 0 1 1 1 2 1 0 1 1 133 0 1 1 5 0 2 1 %
5 0 2 1 15 0 2 1 8 0 2 1 8 0 2 1 8 0 2 1 8 0 1 1 1 4 2 0 1 1 6 0 1 2 1 %
0 1 3 8 0 1 5 2 1}
\Sconcordance{concordance:PracticasResueltas.tex:PracticasResueltas.Rnw:ofs 1971:%
25}
\Sconcordance{concordance:PracticasResueltas.tex:./practica11.Rnw:ofs 1972:%
1 1 2 6 0 1 1 5 0 1 1 12 0 1 1 1 3 1 0 1 2 9 0 1 3 13 0 1 3 10 0 1 3 12 %
0 1 1 7 0 1 1 7 0 1 5 3 0 1 3 6 0 2 2 22 0 2 2 18 0 1 5 2 1}
\Sconcordance{concordance:PracticasResueltas.tex:PracticasResueltas.Rnw:ofs 2124:%
27}
\Sconcordance{concordance:PracticasResueltas.tex:./practica13.Rnw:ofs 2125:%
1 1 1 1 2 6 0 1 1 5 0 1 1 7 0 1 2 6 0 1 1 5 0 1 3 8 0 1 3 6 0 1 3 6 0 1 %
4 17 0 1 3 6 0 1 3 6 0 1 13 11 0 1 1 1 3 9 0 1 3 16 0 1 9 2 1}
\Sconcordance{concordance:PracticasResueltas.tex:PracticasResueltas.Rnw:ofs 2259:%
29}
\Sconcordance{concordance:PracticasResueltas.tex:./practica14.Rnw:ofs 2260:%
1 1 2 6 0 1 1 1 2 5 0 1 2 1 3 6 0 1 3 6 0 1 2 1 3 5 0 1 4 7 0 1 3 1 0 2 %
1 5 0 1 2 1 3 1 0 1 5 3 0 1 4 2 0 1 3 1 0 1 2 2 1 8 0 1 7 1 1}
\Sconcordance{concordance:PracticasResueltas.tex:PracticasResueltas.Rnw:ofs 2339:%
31}
\Sconcordance{concordance:PracticasResueltas.tex:./practica1516.Rnw:ofs 2340:%
3 2 0 3 1 5 0 1 2 1 0 2 1 5 0 1 2 6 0 1 3 2 0 3 1 5 0 1 2 1 0 2 1 5 0 1 %
2 1 0 5 1 5 0 1 3 2 0 2 1 5 0 1 2 1 0 2 1 5 0 1 2 6 0 1 2 1 0 2 1 5 0 1 %
2 6 0 1 2 6 0 1 2 6 0 1 5 4 0 3 1 17 0 1 3 1 0 1 1 1 3 1 0 1 1 21 0 1 4 %
2 0 1 2 3 0 1 2}
\Sconcordance{concordance:PracticasResueltas.tex:PracticasResueltas.Rnw:ofs 2515:%
33}
\Sconcordance{concordance:PracticasResueltas.tex:./practica17.Rnw:ofs 2516:%
1 3 1 1 22 20 0 2 1 5 0 1 1 55 0 1 1 5 0 1 3 2 0 1 1 3 0 1 2 2 1}
\Sconcordance{concordance:PracticasResueltas.tex:PracticasResueltas.Rnw:ofs 2620:%
35}
\Sconcordance{concordance:PracticasResueltas.tex:./practica18.Rnw:ofs 2621:%
1 1 1 1 13 12 0 2 1 12 0 1 2 2 1}
\Sconcordance{concordance:PracticasResueltas.tex:PracticasResueltas.Rnw:ofs 2653:%
37}
\Sconcordance{concordance:PracticasResueltas.tex:./practica21.Rnw:ofs 2654:%
1 8 1 1 3 2 0 1 1 3 0 1 2 3 1 1 2 4 0 1 2 3 1 1 2 4 0 1 2 4 1 1 2 1 0 1 %
1 8 0 1 2 1 1 10 0 1 2 1 1 1 3 2 0 1 2 2 1 1 2 1 1 8 0 1 2 1 1 10 0 1 2 %
3 1}
\Sconcordance{concordance:PracticasResueltas.tex:PracticasResueltas.Rnw:ofs 2750:%
39}
\Sconcordance{concordance:PracticasResueltas.tex:./practica22.Rnw:ofs 2751:%
1 5 1 1 49 48 0 1 2 10 0 1 1 10 0 1 1 11 0 1 2 14 0 1 1 14 0 1 1 15 0 1 %
2 4 1 1 3 2 0 1 39 36 0 1 2 5 0 1 2 15 0 1 2 2 1}
\Sconcordance{concordance:PracticasResueltas.tex:PracticasResueltas.Rnw:ofs 2956:%
41}
\Sconcordance{concordance:PracticasResueltas.tex:./practica23.Rnw:ofs 2957:%
1 4 1 1 6 5 0 1 2 1 0 1 1 14 0 1 5 4 0 2 1 8 0 1 1 8 0 1 1 9 0 1 1 9 0 %
1 1 14 0 1 6 5 0 2 1 14 0 1 8 7 0 1 3 2 0 1 1 14 0 1 4 23 0 1 5 3 1}
\Sconcordance{concordance:PracticasResueltas.tex:PracticasResueltas.Rnw:ofs 3120:%
43}
\Sconcordance{concordance:PracticasResueltas.tex:./practica24.Rnw:ofs 3121:%
1 4 1 1 8 7 0 2 1 32 0 2 1 7 0 1 6 3 1}
\Sconcordance{concordance:PracticasResueltas.tex:PracticasResueltas.Rnw:ofs 3181:%
45}
\Sconcordance{concordance:PracticasResueltas.tex:./practica2526.Rnw:ofs 3182:%
1 1 2 1 0 1 1 6 0 2 1 6 0 2 1 6 0 2 1 25 0 2 1 10 0 1 3 1 1 6 0 2 1 6 0 %
2 1 7 0 2 1 37 0 2 1 15 0 1 3}
\Sconcordance{concordance:PracticasResueltas.tex:PracticasResueltas.Rnw:ofs 3329:%
47 2 1}


\textbf{Practica 2.}



\begin{Schunk}
\begin{Sinput}
> v <- numeric(3);v
\end{Sinput}
\begin{Soutput}
[1] 0 0 0
\end{Soutput}
\begin{Sinput}
> v[3] <- 17; v
\end{Sinput}
\begin{Soutput}
[1]  0  0 17
\end{Soutput}
\begin{Sinput}
> x <- c(2, 4, 3.1, 8, 6)
> x
\end{Sinput}
\begin{Soutput}
[1] 2.0 4.0 3.1 8.0 6.0
\end{Soutput}
\begin{Sinput}
> is.integer(x) 
\end{Sinput}
\begin{Soutput}
[1] FALSE
\end{Soutput}
\begin{Sinput}
> is.double(x)
\end{Sinput}
\begin{Soutput}
[1] TRUE
\end{Soutput}
\begin{Sinput}
> length(x)
\end{Sinput}
\begin{Soutput}
[1] 5
\end{Soutput}
\begin{Sinput}
> x <- edit(x)
> y = 1:4; y
\end{Sinput}
\begin{Soutput}
[1] 1 2 3 4
\end{Soutput}
\begin{Sinput}
> y[2] <- 5
> u <- 1:12
> u
\end{Sinput}
\begin{Soutput}
 [1]  1  2  3  4  5  6  7  8  9 10 11 12
\end{Soutput}
\begin{Sinput}
> u1=u[2 * 1:5]
> u1
\end{Sinput}
\begin{Soutput}
[1]  2  4  6  8 10
\end{Soutput}
\begin{Sinput}
> assign("z", c(x, 0, x))
> z
\end{Sinput}
\begin{Soutput}
 [1] 2.0 4.0 3.1 8.0 6.0 0.0 2.0 4.0 3.1 8.0 6.0
\end{Soutput}
\begin{Sinput}
> s1 <- seq(2, 10); s1 
\end{Sinput}
\begin{Soutput}
[1]  2  3  4  5  6  7  8  9 10
\end{Soutput}
\begin{Sinput}
> s2 = seq(from=-1, to=5); s2
\end{Sinput}
\begin{Soutput}
[1] -1  0  1  2  3  4  5
\end{Soutput}
\begin{Sinput}
> s3<-seq(to=2, from=-2); s3
\end{Sinput}
\begin{Soutput}
[1] -2 -1  0  1  2
\end{Soutput}
\begin{Sinput}
> s4=seq(from=-3, to=3, by=0.2); s4
\end{Sinput}
\begin{Soutput}
 [1] -3.0 -2.8 -2.6 -2.4 -2.2 -2.0 -1.8 -1.6 -1.4 -1.2 -1.0 -0.8 -0.6 -0.4 -0.2
[16]  0.0  0.2  0.4  0.6  0.8  1.0  1.2  1.4  1.6  1.8  2.0  2.2  2.4  2.6  2.8
[31]  3.0
\end{Soutput}
\begin{Sinput}
> s5 <- rep(s3, times=3); s5
\end{Sinput}
\begin{Soutput}
 [1] -2 -1  0  1  2 -2 -1  0  1  2 -2 -1  0  1  2
\end{Soutput}
\begin{Sinput}
> 1/x
\end{Sinput}
\begin{Soutput}
[1] 0.5000000 0.2500000 0.3225806 0.1250000 0.1666667
\end{Soutput}
\begin{Sinput}
> v=2*x+z+1
> v
\end{Sinput}
\begin{Soutput}
 [1]  7.0 13.0 10.3 25.0 19.0  5.0 11.0 11.2 20.1 21.0 11.0
\end{Soutput}
\begin{Sinput}
> e<- c(1, 2, 3, 4); e2<-c(4, 5, 6, 7); crossprod(e, e2) 
\end{Sinput}
\begin{Soutput}
     [,1]
[1,]   60
\end{Soutput}
\begin{Sinput}
> xt = t(x)
> xt
\end{Sinput}
\begin{Soutput}
     [,1] [,2] [,3] [,4] [,5]
[1,]    2    4  3.1    8    6
\end{Soutput}
\begin{Sinput}
> u = exp(y);u
\end{Sinput}
\begin{Soutput}
[1]   2.718282 148.413159  20.085537  54.598150
\end{Soutput}
\begin{Sinput}
> options(digits=10); u
\end{Sinput}
\begin{Soutput}
[1]   2.718281828 148.413159103  20.085536923  54.598150033
\end{Soutput}
\begin{Sinput}
> resum <- c(length(y),sum(y), prod(y), min(y), max(y)); resum
\end{Sinput}
\begin{Soutput}
[1]  4 13 60  1  5
\end{Soutput}
\begin{Sinput}
> yo <- sort(y); yo
\end{Sinput}
\begin{Soutput}
[1] 1 3 4 5
\end{Soutput}
\begin{Sinput}
> deptos <- c("Santa Ana", "Sonsonate", "San Salvador"); deptos
\end{Sinput}
\begin{Soutput}
[1] "Santa Ana"    "Sonsonate"    "San Salvador"
\end{Soutput}
\begin{Sinput}
> deptos[4]="Ahuachapán"; deptos
\end{Sinput}
\begin{Soutput}
[1] "Santa Ana"    "Sonsonate"    "San Salvador" "Ahuachapán"  
\end{Soutput}
\begin{Sinput}
> codDeptos <- c(11, 12, 13, 14)
> Oriente <- codDeptos [c("La Unión", "San Miguel")];Oriente
\end{Sinput}
\begin{Soutput}
[1] NA NA
\end{Soutput}
\begin{Sinput}
> etiqs<-paste(c("X", "Y"), 1:10, sep=""); etiqs
\end{Sinput}
\begin{Soutput}
 [1] "X1"  "Y2"  "X3"  "Y4"  "X5"  "Y6"  "X7"  "Y8"  "X9"  "Y10"
\end{Soutput}
\begin{Sinput}
> M <- matrix(numeric(), nrow = 3, ncol=4)
> M[2,3] <- 6
> M
\end{Sinput}
\begin{Soutput}
     [,1] [,2] [,3] [,4]
[1,]   NA   NA   NA   NA
[2,]   NA   NA    6   NA
[3,]   NA   NA   NA   NA
\end{Soutput}
\begin{Sinput}
> A <- matrix(c(2, 4, 6, 8, 10, 12), nrow=2, ncol=3)
> A
\end{Sinput}
\begin{Soutput}
     [,1] [,2] [,3]
[1,]    2    6   10
[2,]    4    8   12
\end{Soutput}
\begin{Sinput}
> mode(A)
\end{Sinput}
\begin{Soutput}
[1] "numeric"
\end{Soutput}
\begin{Sinput}
> dim(A)
\end{Sinput}
\begin{Soutput}
[1] 2 3
\end{Soutput}
\begin{Sinput}
> attributes(A)
\end{Sinput}
\begin{Soutput}
$dim
[1] 2 3
\end{Soutput}
\begin{Sinput}
> is.matrix(A)
\end{Sinput}
\begin{Soutput}
[1] TRUE
\end{Soutput}
\begin{Sinput}
> is.array(A)
\end{Sinput}
\begin{Soutput}
[1] TRUE
\end{Soutput}
\begin{Sinput}
> B <- matrix(1:12, nrow=3, ncol=4)
> B
\end{Sinput}
\begin{Soutput}
     [,1] [,2] [,3] [,4]
[1,]    1    4    7   10
[2,]    2    5    8   11
[3,]    3    6    9   12
\end{Soutput}
\begin{Sinput}
> x1 <- seq(0, 10, 2); x1
\end{Sinput}
\begin{Soutput}
[1]  0  2  4  6  8 10
\end{Soutput}
\begin{Sinput}
> x2 <- seq(1, 11, 2); x2
\end{Sinput}
\begin{Soutput}
[1]  1  3  5  7  9 11
\end{Soutput}
\begin{Sinput}
> x3 <- runif(6); x3
\end{Sinput}
\begin{Soutput}
[1] 0.2043420891 0.2013161827 0.7850665464 0.2219624307 0.4837569490
[6] 0.5620540997
\end{Soutput}
\begin{Sinput}
> Xcol <- cbind(x1, x2, x3); Xcol
\end{Sinput}
\begin{Soutput}
     x1 x2           x3
[1,]  0  1 0.2043420891
[2,]  2  3 0.2013161827
[3,]  4  5 0.7850665464
[4,]  6  7 0.2219624307
[5,]  8  9 0.4837569490
[6,] 10 11 0.5620540997
\end{Soutput}
\begin{Sinput}
> Xfil <- rbind(x1, x2, x3); Xfil
\end{Sinput}
\begin{Soutput}
           [,1]         [,2]         [,3]         [,4]        [,5]
x1 0.0000000000 2.0000000000 4.0000000000 6.0000000000 8.000000000
x2 1.0000000000 3.0000000000 5.0000000000 7.0000000000 9.000000000
x3 0.2043420891 0.2013161827 0.7850665464 0.2219624307 0.483756949
            [,6]
x1 10.0000000000
x2 11.0000000000
x3  0.5620540997
\end{Soutput}
\begin{Sinput}
> X <- Xfil[1:3, c(2, 3)]; X
\end{Sinput}
\begin{Soutput}
           [,1]         [,2]
x1 2.0000000000 4.0000000000
x2 3.0000000000 5.0000000000
x3 0.2013161827 0.7850665464
\end{Soutput}
\begin{Sinput}
> v<-c(1, 2); v %*%A
\end{Sinput}
\begin{Soutput}
     [,1] [,2] [,3]
[1,]   10   22   34
\end{Soutput}
\begin{Sinput}
> P <- A %*% B; P
\end{Sinput}
\begin{Soutput}
     [,1] [,2] [,3] [,4]
[1,]   44   98  152  206
[2,]   56  128  200  272
\end{Soutput}
\begin{Sinput}
> 2*A
\end{Sinput}
\begin{Soutput}
     [,1] [,2] [,3]
[1,]    4   12   20
[2,]    8   16   24
\end{Soutput}
\begin{Sinput}
> length(A)
\end{Sinput}
\begin{Soutput}
[1] 6
\end{Soutput}
\begin{Sinput}
> T=sqrt(B); T
\end{Sinput}
\begin{Soutput}
            [,1]        [,2]        [,3]        [,4]
[1,] 1.000000000 2.000000000 2.645751311 3.162277660
[2,] 1.414213562 2.236067977 2.828427125 3.316624790
[3,] 1.732050808 2.449489743 3.000000000 3.464101615
\end{Soutput}
\begin{Sinput}
> t(A)
\end{Sinput}
\begin{Soutput}
     [,1] [,2]
[1,]    2    4
[2,]    6    8
[3,]   10   12
\end{Soutput}
\begin{Sinput}
> C <- matrix(c(2, 1, 10, 12), nrow=2, ncol=2); C
\end{Sinput}
\begin{Soutput}
     [,1] [,2]
[1,]    2   10
[2,]    1   12
\end{Soutput}
\begin{Sinput}
> det(C)
\end{Sinput}
\begin{Soutput}
[1] 14
\end{Soutput}
\begin{Sinput}
> InvC <- solve(C)
> eigen(C)
\end{Sinput}
\begin{Soutput}
$values
[1] 12.916079783  1.083920217

$vectors
              [,1]           [,2]
[1,] -0.6754894393 -0.99583021557
[2,] -0.7373696613  0.09122599279
\end{Soutput}
\begin{Sinput}
> c(length(A), sum(A), prod(A), min(A), max(A))
\end{Sinput}
\begin{Soutput}
[1]     6    42 46080     2    12
\end{Soutput}
\begin{Sinput}
> nombres <- matrix(c("Carlos", "José", "Caren", "René", "Mar??a", "Mario"),
+                   nrow=3, ncol=2); nombres
\end{Sinput}
\begin{Soutput}
     [,1]     [,2]    
[1,] "Carlos" "René"  
[2,] "José"   "Mar??a"
[3,] "Caren"  "Mario" 
\end{Soutput}
\begin{Sinput}
> X <- array(c(1, 3, 5, 7, 9, 11), dim=c(2, 3)); X
\end{Sinput}
\begin{Soutput}
     [,1] [,2] [,3]
[1,]    1    5    9
[2,]    3    7   11
\end{Soutput}
\begin{Sinput}
> Z <- array(1, c(3, 3)); Z
\end{Sinput}
\begin{Soutput}
     [,1] [,2] [,3]
[1,]    1    1    1
[2,]    1    1    1
[3,]    1    1    1
\end{Soutput}
\begin{Sinput}
> W <- 2*Z+1
> W
\end{Sinput}
\begin{Soutput}
     [,1] [,2] [,3]
[1,]    3    3    3
[2,]    3    3    3
[3,]    3    3    3
\end{Soutput}
\begin{Sinput}
> TX <- t(X)
> TX
\end{Sinput}
\begin{Soutput}
     [,1] [,2]
[1,]    1    3
[2,]    5    7
[3,]    9   11
\end{Soutput}
\begin{Sinput}
> a <- c(2, 4, 6)
> a
\end{Sinput}
\begin{Soutput}
[1] 2 4 6
\end{Soutput}
\begin{Sinput}
> b <- 1:3
> b
\end{Sinput}
\begin{Soutput}
[1] 1 2 3
\end{Soutput}
\begin{Sinput}
> M <- a %o% b
> M 
\end{Sinput}
\begin{Soutput}
     [,1] [,2] [,3]
[1,]    2    4    6
[2,]    4    8   12
[3,]    6   12   18
\end{Soutput}
\begin{Sinput}
> Arreglo3 <- array(c(1:8, 11:18, 111:118), dim = c(2, 4, 3))
> Arreglo3
\end{Sinput}
\begin{Soutput}
, , 1

     [,1] [,2] [,3] [,4]
[1,]    1    3    5    7
[2,]    2    4    6    8

, , 2

     [,1] [,2] [,3] [,4]
[1,]   11   13   15   17
[2,]   12   14   16   18

, , 3

     [,1] [,2] [,3] [,4]
[1,]  111  113  115  117
[2,]  112  114  116  118
\end{Soutput}
\begin{Sinput}
> 
> 
> 
> 
> 
> 
> 
> 
> 
> 
> 
\end{Sinput}
\end{Schunk}

\textbf{Practica 3.}


\begin{Schunk}
\begin{Sinput}
> sexo <- c("M", "F", "F", "M", "F", "F", "M")
> sexo
\end{Sinput}
\begin{Soutput}
[1] "M" "F" "F" "M" "F" "F" "M"
\end{Soutput}
\begin{Sinput}
> edad <- c(19, 20, 19, 22, 20, 21, 19)
> edad
\end{Sinput}
\begin{Soutput}
[1] 19 20 19 22 20 21 19
\end{Soutput}
\begin{Sinput}
> FactorSexo = factor(sexo)
> FactorSexo
\end{Sinput}
\begin{Soutput}
[1] M F F M F F M
Levels: F M
\end{Soutput}
\begin{Sinput}
> mediaEdad <- tapply(edad, FactorSexo, mean) 
> mediaEdad
\end{Sinput}
\begin{Soutput}
 F  M 
20 20 
\end{Soutput}
\begin{Sinput}
> is.vector(mediaEdad)
\end{Sinput}
\begin{Soutput}
[1] FALSE
\end{Soutput}
\begin{Sinput}
> is.matrix(mediaEdad)
\end{Sinput}
\begin{Soutput}
[1] FALSE
\end{Soutput}
\begin{Sinput}
> is.list(mediaEdad)
\end{Sinput}
\begin{Soutput}
[1] FALSE
\end{Soutput}
\begin{Sinput}
> is.table(mediaEdad)
\end{Sinput}
\begin{Soutput}
[1] FALSE
\end{Soutput}
\begin{Sinput}
> is.array(mediaEdad)
\end{Sinput}
\begin{Soutput}
[1] TRUE
\end{Soutput}
\begin{Sinput}
> factor()
\end{Sinput}
\begin{Soutput}
factor(0)
Levels: 
\end{Soutput}
\begin{Sinput}
> lista1<-list(padre="Pedro", madre="Mar?a", no.hijos=3, edad.hijos=c(4,7,9))
> lista1
\end{Sinput}
\begin{Soutput}
$padre
[1] "Pedro"

$madre
[1] "Mar?a"

$no.hijos
[1] 3

$edad.hijos
[1] 4 7 9
\end{Soutput}
\begin{Sinput}
> is.matrix(lista1)
\end{Sinput}
\begin{Soutput}
[1] FALSE
\end{Soutput}
\begin{Sinput}
> is.vector(lista1$edad.hijos)
\end{Sinput}
\begin{Soutput}
[1] TRUE
\end{Soutput}
\begin{Sinput}
> lista1["madre"]
\end{Sinput}
\begin{Soutput}
$madre
[1] "Mar?a"
\end{Soutput}
\begin{Sinput}
> lista1[[4]][2]
\end{Sinput}
\begin{Soutput}
[1] 7
\end{Soutput}
\begin{Sinput}
> lista1["padre"]
\end{Sinput}
\begin{Soutput}
$padre
[1] "Pedro"
\end{Soutput}
\begin{Sinput}
> lista1$padre
\end{Sinput}
\begin{Soutput}
[1] "Pedro"
\end{Soutput}
\begin{Sinput}
> lista1$edad.hijos[2]
\end{Sinput}
\begin{Soutput}
[1] 7
\end{Soutput}
\begin{Sinput}
> lista1[[4]][2]
\end{Sinput}
\begin{Soutput}
[1] 7
\end{Soutput}
\begin{Sinput}
> lista1[["pedro"]]
\end{Sinput}
\begin{Soutput}
NULL
\end{Soutput}
\begin{Sinput}
> x <- "nombre"; lista1[x]
\end{Sinput}
\begin{Soutput}
$<NA>
NULL
\end{Soutput}
\begin{Sinput}
> subLista <- lista1[4]; subLista
\end{Sinput}
\begin{Soutput}
$edad.hijos
[1] 4 7 9
\end{Soutput}
\begin{Sinput}
> lista1[5] <- list(sexo.hijos=c("F", "M", "F")); lista1
\end{Sinput}
\begin{Soutput}
$padre
[1] "Pedro"

$madre
[1] "Mar?a"

$no.hijos
[1] 3

$edad.hijos
[1] 4 7 9

[[5]]
[1] "F" "M" "F"
\end{Soutput}
\begin{Sinput}
> lista1 <- edit(lista1)
> S <- matrix(c(3, -sqrt(2), -sqrt(2), 2), nrow=2, ncol=2);S
\end{Sinput}
\begin{Soutput}
             [,1]         [,2]
[1,]  3.000000000 -1.414213562
[2,] -1.414213562  2.000000000
\end{Soutput}
\begin{Sinput}
> autovS <- eigen(S); autovS
\end{Sinput}
\begin{Soutput}
$values
[1] 4 1

$vectors
              [,1]          [,2]
[1,] -0.8164965809 -0.5773502692
[2,]  0.5773502692 -0.8164965809
\end{Soutput}
\begin{Sinput}
> evals <- eigen(S)$values; evals
\end{Sinput}
\begin{Soutput}
[1] 4 1
\end{Soutput}
\begin{Sinput}
> Notas <- matrix(c(2, 5, 7, 6, 8, 2, 4, 9, 10), ncol=3,
+                 dimnames=list(c("Matem?tica","?lgebra","Geometr?a"),
+                               c("Juan","Jos?","Ren?"))); Notas
\end{Sinput}
\begin{Soutput}
           Juan Jos? Ren?
Matem?tica    2    6    4
?lgebra       5    8    9
Geometr?a     7    2   10
\end{Soutput}
\begin{Sinput}
> ncol=(3)
> log <- sample(c(TRUE, FALSE), size = 20, replace = TRUE)
> log
\end{Sinput}
\begin{Soutput}
 [1]  TRUE  TRUE  TRUE FALSE  TRUE FALSE  TRUE  TRUE FALSE FALSE  TRUE  TRUE
[13] FALSE FALSE FALSE FALSE  TRUE  TRUE FALSE FALSE
\end{Soutput}
\begin{Sinput}
> comp <- rnorm(20) + runif(20) * (1i)
> comp
\end{Sinput}
\begin{Soutput}
 [1]  0.5601797984+0.6126686754i -0.3862760686+0.4215028898i
 [3]  0.2419075811+0.7194944893i  0.2636025214+0.4969554804i
 [5]  0.7911994451+0.4012579382i  1.5548481267+0.7451974021i
 [7] -1.0973340557+0.2529172190i -0.4333767334+0.3393570352i
 [9] -0.2032159801+0.3020620167i  0.4808794229+0.0219908867i
[11] -0.5686133538+0.4086344237i -0.4786282477+0.1328211075i
[13]  0.5711407490+0.8532294447i  0.0780557282+0.8639678934i
[15]  0.2834984722+0.3397298041i  0.4218138908+0.8335026775i
[17] -1.2174206547+0.8842295588i -1.1995075297+0.5758123698i
[19]  1.4800057073+0.1985752415i -0.9227081935+0.6267867342i
\end{Soutput}
\begin{Sinput}
> num <- rnorm(20, mean=0, sd=1)
> num
\end{Sinput}
\begin{Soutput}
 [1] -0.18707244590 -1.43943460126  0.06882203842  0.80341760171 -1.39671712959
 [6] -0.06853830305  1.51350927755  0.14105305409  2.46610007809 -0.13691818445
[11] -1.40967884930 -0.75857361493  2.44742836993 -0.73890316661  0.02818394939
[16] -1.88340939917 -0.84438481163 -0.09098792099 -0.26556850080 -0.95222365820
\end{Soutput}
\begin{Sinput}
> df1 <- data.frame(log, comp, num)
> df1
\end{Sinput}
\begin{Soutput}
     log                        comp            num
1   TRUE  0.5601797984+0.6126686754i -0.18707244590
2   TRUE -0.3862760686+0.4215028898i -1.43943460126
3   TRUE  0.2419075811+0.7194944893i  0.06882203842
4  FALSE  0.2636025214+0.4969554804i  0.80341760171
5   TRUE  0.7911994451+0.4012579382i -1.39671712959
6  FALSE  1.5548481267+0.7451974021i -0.06853830305
7   TRUE -1.0973340557+0.2529172190i  1.51350927755
8   TRUE -0.4333767334+0.3393570352i  0.14105305409
9  FALSE -0.2032159801+0.3020620167i  2.46610007809
10 FALSE  0.4808794229+0.0219908867i -0.13691818445
11  TRUE -0.5686133538+0.4086344237i -1.40967884930
12  TRUE -0.4786282477+0.1328211075i -0.75857361493
13 FALSE  0.5711407490+0.8532294447i  2.44742836993
14 FALSE  0.0780557282+0.8639678934i -0.73890316661
15 FALSE  0.2834984722+0.3397298041i  0.02818394939
16 FALSE  0.4218138908+0.8335026775i -1.88340939917
17  TRUE -1.2174206547+0.8842295588i -0.84438481163
18  TRUE -1.1995075297+0.5758123698i -0.09098792099
19 FALSE  1.4800057073+0.1985752415i -0.26556850080
20 FALSE -0.9227081935+0.6267867342i -0.95222365820
\end{Soutput}
\begin{Sinput}
> nombres <- c("logico", "complejo", "numerico")
> names(df1) <- nombres; df1
\end{Sinput}
\begin{Soutput}
   logico                    complejo       numerico
1    TRUE  0.5601797984+0.6126686754i -0.18707244590
2    TRUE -0.3862760686+0.4215028898i -1.43943460126
3    TRUE  0.2419075811+0.7194944893i  0.06882203842
4   FALSE  0.2636025214+0.4969554804i  0.80341760171
5    TRUE  0.7911994451+0.4012579382i -1.39671712959
6   FALSE  1.5548481267+0.7451974021i -0.06853830305
7    TRUE -1.0973340557+0.2529172190i  1.51350927755
8    TRUE -0.4333767334+0.3393570352i  0.14105305409
9   FALSE -0.2032159801+0.3020620167i  2.46610007809
10  FALSE  0.4808794229+0.0219908867i -0.13691818445
11   TRUE -0.5686133538+0.4086344237i -1.40967884930
12   TRUE -0.4786282477+0.1328211075i -0.75857361493
13  FALSE  0.5711407490+0.8532294447i  2.44742836993
14  FALSE  0.0780557282+0.8639678934i -0.73890316661
15  FALSE  0.2834984722+0.3397298041i  0.02818394939
16  FALSE  0.4218138908+0.8335026775i -1.88340939917
17   TRUE -1.2174206547+0.8842295588i -0.84438481163
18   TRUE -1.1995075297+0.5758123698i -0.09098792099
19  FALSE  1.4800057073+0.1985752415i -0.26556850080
20  FALSE -0.9227081935+0.6267867342i -0.95222365820
\end{Soutput}
\begin{Sinput}
> row.names(df1) <- letters[1:20]
> df1
\end{Sinput}
\begin{Soutput}
  logico                    complejo       numerico
a   TRUE  0.5601797984+0.6126686754i -0.18707244590
b   TRUE -0.3862760686+0.4215028898i -1.43943460126
c   TRUE  0.2419075811+0.7194944893i  0.06882203842
d  FALSE  0.2636025214+0.4969554804i  0.80341760171
e   TRUE  0.7911994451+0.4012579382i -1.39671712959
f  FALSE  1.5548481267+0.7451974021i -0.06853830305
g   TRUE -1.0973340557+0.2529172190i  1.51350927755
h   TRUE -0.4333767334+0.3393570352i  0.14105305409
i  FALSE -0.2032159801+0.3020620167i  2.46610007809
j  FALSE  0.4808794229+0.0219908867i -0.13691818445
k   TRUE -0.5686133538+0.4086344237i -1.40967884930
l   TRUE -0.4786282477+0.1328211075i -0.75857361493
m  FALSE  0.5711407490+0.8532294447i  2.44742836993
n  FALSE  0.0780557282+0.8639678934i -0.73890316661
o  FALSE  0.2834984722+0.3397298041i  0.02818394939
p  FALSE  0.4218138908+0.8335026775i -1.88340939917
q   TRUE -1.2174206547+0.8842295588i -0.84438481163
r   TRUE -1.1995075297+0.5758123698i -0.09098792099
s  FALSE  1.4800057073+0.1985752415i -0.26556850080
t  FALSE -0.9227081935+0.6267867342i -0.95222365820
\end{Soutput}
\begin{Sinput}
> edad <- c(18, 21, 45, 54); edad
\end{Sinput}
\begin{Soutput}
[1] 18 21 45 54
\end{Soutput}
\begin{Sinput}
> datos <- matrix(c(150, 160, 180, 205, 65, 68, 65, 69), ncol=2, dimnames=list(c(),
+                                                                              c("Estatura","Peso"))); datos
\end{Sinput}
\begin{Soutput}
     Estatura Peso
[1,]      150   65
[2,]      160   68
[3,]      180   65
[4,]      205   69
\end{Soutput}
\begin{Sinput}
> sexo <- c("F", "M", "M", "M"); sexo
\end{Sinput}
\begin{Soutput}
[1] "F" "M" "M" "M"
\end{Soutput}
\begin{Sinput}
> hoja1 <- data.frame(Edad=edad, datos, Sexo=sexo)
> hoja1
\end{Sinput}
\begin{Soutput}
  Edad Estatura Peso Sexo
1   18      150   65    F
2   21      160   68    M
3   45      180   65    M
4   54      205   69    M
\end{Soutput}
\begin{Sinput}
> search()
\end{Sinput}
\begin{Soutput}
[1] ".GlobalEnv"        "package:stats"     "package:graphics" 
[4] "package:grDevices" "package:utils"     "package:datasets" 
[7] "package:methods"   "Autoloads"         "package:base"     
\end{Soutput}
\begin{Sinput}
> attach(hoja1)
> search()
\end{Sinput}
\begin{Soutput}
 [1] ".GlobalEnv"        "hoja1"             "package:stats"    
 [4] "package:graphics"  "package:grDevices" "package:utils"    
 [7] "package:datasets"  "package:methods"   "Autoloads"        
[10] "package:base"     
\end{Soutput}
\begin{Sinput}
> Edad
\end{Sinput}
\begin{Soutput}
[1] 18 21 45 54
\end{Soutput}
\begin{Sinput}
> hoja1$Peso <- Peso+1
> hoja1
\end{Sinput}
\begin{Soutput}
  Edad Estatura Peso Sexo
1   18      150   66    F
2   21      160   69    M
3   45      180   66    M
4   54      205   70    M
\end{Soutput}
\begin{Sinput}
> detach(hoja1)
> edad
\end{Sinput}
\begin{Soutput}
[1] 18 21 45 54
\end{Soutput}
\begin{Sinput}
> 
> 
> 
\end{Sinput}
\end{Schunk}

\textbf{Practica 4.}



Entrada1 <- read.table("datos01.txt", header=TRUE)
Entrada1
Edat1 <- scan("datos01.txt", list(X1=0, X2=0), skip = 1, flush = TRUE, quiet = TRUE)
Edat1
pp <- scan("datos02.txt", skip = 1, quiet= TRUE)
pp

library(foreign)
baseproductos <-read.table("productos.csv",header=TRUE,sep = ",")
baseproductos

library(Hmisc)
Baseimportante<-spss.get("Mundo.sav",use.value.labels =TRUE)
Baseimportante




\textbf{Practica 5.}


\begin{Schunk}
\begin{Sinput}
> x <- c(6:10)
> x
\end{Sinput}
\begin{Soutput}
[1]  6  7  8  9 10
\end{Soutput}
\begin{Sinput}
> sqrt(x) 
\end{Sinput}
\begin{Soutput}
[1] 2.449489743 2.645751311 2.828427125 3.000000000 3.162277660
\end{Soutput}
\begin{Sinput}
> sqrt(ifelse(x >= 0, x, NA))
\end{Sinput}
\begin{Soutput}
[1] 2.449489743 2.645751311 2.828427125 3.000000000 3.162277660
\end{Soutput}
\begin{Sinput}
> x <- c(2, 6, 4, 7, 5, 1)
> x
\end{Sinput}
\begin{Soutput}
[1] 2 6 4 7 5 1
\end{Soutput}
\begin{Sinput}
> suma<-0; for(i in 1:3) suma = suma+x[i]
> suma
\end{Sinput}
\begin{Soutput}
[1] 12
\end{Soutput}
\begin{Sinput}
> media <- function(x)
+ {
+ n = length(x)
+ suma <- 0.0
+ for(i in 1:n) suma = suma + x[i]
+ media = suma/n
+ }
> func.cuadratica <- function(x)
+ {
+ 3*x^2-5*x+2
+ }
> y <- func.cuadratica(2)
> y
\end{Sinput}
\begin{Soutput}
[1] 4
\end{Soutput}
\begin{Sinput}
> save(media, file= "media.RData")
> rm(list=ls(all=TRUE))
> load("media.RData")
> media <- function(x)
+ {
+ n = length(x)
+ suma <- 0.0
+ for(i in 1:n) suma = suma + x[i]
+ media = suma/n
+ }
> x <- 1:5
> (media(x))
\end{Sinput}
\begin{Soutput}
[1] 3
\end{Soutput}
\begin{Sinput}
> y <- c(5, 8 , 4, 9)
> (media(y))
\end{Sinput}
\begin{Soutput}
[1] 6.5
\end{Soutput}
\begin{Sinput}
> Seno <- function(x)
+ {
+ y = sin(x)
+ plot(x, y, main="Ejemplo de gr?ficos en R",
+ xlab="x", ylab="y = Seno(x)", col="blue", pch=1)
+ }
> x<-seq(-pi, pi, len=100)
> Seno(x)
> func.cuadratica <- function(x)
+ {
+ 3*x^2-5*x+2
+ }
> y <- func.cuadratica(2)
> y
\end{Sinput}
\begin{Soutput}
[1] 4
\end{Soutput}
\begin{Sinput}
> media <- function(x)
+ {
+ n = length(x)
+ suma <- 0.0
+ for(i in 1:n) suma = suma + x[i]
+ media = suma/n
+ }
> 
\end{Sinput}
\end{Schunk}
\begin{Schunk}
\begin{Sinput}
> library(splines)
> library( RcmdrMisc)
> library(car)
> library(sandwich)
> library(relimp, pos=15)
> 
> 
> 
\end{Sinput}
\end{Schunk}

  
\textbf{Practica 6.}


\begin{Schunk}
\begin{Sinput}
> #"CC"=Coca_Cola
> #"PC"=Pepsi_Cola
> #"SC"=Salva_Cola
> Tipo<-c("CC","PC","SC");Tipo
\end{Sinput}
\begin{Soutput}
[1] "CC" "PC" "SC"
\end{Soutput}
\begin{Sinput}
> Consumo<-sample(Tipo,20,replace=TRUE);Consumo
\end{Sinput}
\begin{Soutput}
 [1] "SC" "CC" "SC" "PC" "CC" "PC" "PC" "PC" "PC" "CC" "SC" "SC" "CC" "PC" "PC"
[16] "SC" "CC" "SC" "CC" "SC"
\end{Soutput}
\begin{Sinput}
> data.entry(Consumo)
> write(Consumo, "Consumo.txt")
> frec <- table(Consumo); frec
\end{Sinput}
\begin{Soutput}
Consumo
CC PC SC 
 6  7  7 
\end{Soutput}
\begin{Sinput}
> prop <- table(Consumo)/length(Consumo); prop
\end{Sinput}
\begin{Soutput}
Consumo
  CC   PC   SC 
0.30 0.35 0.35 
\end{Soutput}
\begin{Sinput}
> summary(Consumo)
\end{Sinput}
\begin{Soutput}
   Length     Class      Mode 
       20 character character 
\end{Soutput}
\begin{Sinput}
> barplot(frec, main="Gráfico de barras", xlab=" Consumo", col=c("yellow", "white", "red"),
+ sub="Agosto-2012")
> barplot(prop, main="Gráfico de barras", xlab=" Consumo\n", col=c("yellow", "white",
+ "red"), sub="Agosto-2012")
> pie(frec, main="Gráfico de pastel", xlab="Tipo de Consumo", col=c("yellow", "white",
+ "cyan"), sub="Agosto-2012")
> names(frec) = c("Coca Cola", "Pepsi", "Salva Cola")
> pie(frec, main="Gráfico de pastel", xlab=" Consumo", radius=1, col=c("red", "gray",
+ "cyan"), sub="Agosto-2012")
> n <- length(frec)
> hoja <- data.frame(frec); hoja
\end{Sinput}
\begin{Soutput}
        Var1 Freq
1  Coca Cola    6
2      Pepsi    7
3 Salva Cola    7
\end{Soutput}
\begin{Sinput}
> etiq <- c(paste(hoja$Var1, "-", hoja$Freq)); etiq
\end{Sinput}
\begin{Soutput}
[1] "Coca Cola - 6"  "Pepsi - 7"      "Salva Cola - 7"
\end{Soutput}
\begin{Sinput}
> pie(frec, main="Gráfico de pastel", labels=etiq, col=rainbow(n), border=TRUE)
\end{Sinput}
\end{Schunk}



\textbf{Practica 7.}


\begin{Schunk}
\begin{Sinput}
> Hijos<-c(2,1,2,1,4,2,3,0,2,3,3,2,1,0,2,4,1,2,1,3,4,1,2,3,1,5,2,3,1,2)
> data.entry(Hijos)
> Hijos
\end{Sinput}
\begin{Soutput}
 [1] 2 1 2 1 4 2 3 0 2 3 3 2 1 0 2 4 1 2 1 3 4 1 2 3 1 5 2 3 1 2
\end{Soutput}
\begin{Sinput}
> length(Hijos)
\end{Sinput}
\begin{Soutput}
[1] 30
\end{Soutput}
\begin{Sinput}
> write(Hijos, "Hijos.txt")
> ls()
\end{Sinput}
\begin{Soutput}
 [1] "Consumo"         "etiq"            "frec"            "func.cuadratica"
 [5] "Hijos"           "hoja"            "media"           "n"              
 [9] "prop"            "Seno"            "Tipo"            "x"              
[13] "y"              
\end{Soutput}
\begin{Sinput}
> rm(list=ls(all=TRUE)); ls()
\end{Sinput}
\begin{Soutput}
character(0)
\end{Soutput}
\begin{Sinput}
> X <- scan("Hijos.txt", what = integer(0), na.strings = "NA", flush=FALSE)
> ls()
\end{Sinput}
\begin{Soutput}
[1] "X"
\end{Soutput}
\begin{Sinput}
> stripchart(X, method="stack", vertical=FALSE, col="blue", pch=1, main="Gráfico de\n
+ puntos", xlab="Número de hijos")
> fab <- table(X); fab
\end{Sinput}
\begin{Soutput}
X
 0  1  2  3  4  5 
 2  8 10  6  3  1 
\end{Soutput}
\begin{Sinput}
> fre <- fab/length(X); fre
\end{Sinput}
\begin{Soutput}
X
            0             1             2             3             4 
0.06666666667 0.26666666667 0.33333333333 0.20000000000 0.10000000000 
            5 
0.03333333333 
\end{Soutput}
\begin{Sinput}
> Fac <- cumsum(fab); Fac
\end{Sinput}
\begin{Soutput}
 0  1  2  3  4  5 
 2 10 20 26 29 30 
\end{Soutput}
\begin{Sinput}
> Far <- Fac/length(X); Far
\end{Sinput}
\begin{Soutput}
            0             1             2             3             4 
0.06666666667 0.33333333333 0.66666666667 0.86666666667 0.96666666667 
            5 
1.00000000000 
\end{Soutput}
\begin{Sinput}
> options(digits=2)
> tabla <- data.frame(fab=fab, fre=fre, Fac=Fac, Far=Far)
> names(tabla) <- c("X", "fab", "free.X", "fre", "Fac", "Far")
> tabla
\end{Sinput}
\begin{Soutput}
  X fab free.X   fre Fac   Far
0 0   2      0 0.067   2 0.067
1 1   8      1 0.267  10 0.333
2 2  10      2 0.333  20 0.667
3 3   6      3 0.200  26 0.867
4 4   3      4 0.100  29 0.967
5 5   1      5 0.033  30 1.000
\end{Soutput}
\begin{Sinput}
> tfre <- data.frame(X=tabla$X, fab=tabla$fab, fre=tabla$fre, Fac=tabla$Fac, Far=tabla$Far)
> tfre
\end{Sinput}
\begin{Soutput}
  X fab   fre Fac   Far
1 0   2 0.067   2 0.067
2 1   8 0.267  10 0.333
3 2  10 0.333  20 0.667
4 3   6 0.200  26 0.867
5 4   3 0.100  29 0.967
6 5   1 0.033  30 1.000
\end{Soutput}
\begin{Sinput}
> media <- mean(X, na.rm = FALSE); media
\end{Sinput}
\begin{Soutput}
[1] 2.1
\end{Soutput}
\begin{Sinput}
> for(i in 1:length(X)) if (fab[i] == max(fab)) break()
> moda <- names(fab[i]); moda # R no tiene incorporada una función para la moda
\end{Sinput}
\begin{Soutput}
[1] "2"
\end{Soutput}
\begin{Sinput}
> mediana <- median(X); mediana
\end{Sinput}
\begin{Soutput}
[1] 2
\end{Soutput}
\begin{Sinput}
> range(X)
\end{Sinput}
\begin{Soutput}
[1] 0 5
\end{Soutput}
\begin{Sinput}
> cuasivar <- var(X); cuasivar
\end{Sinput}
\begin{Soutput}
[1] 1.5
\end{Soutput}
\begin{Sinput}
> s <- sd(X); s
\end{Sinput}
\begin{Soutput}
[1] 1.2
\end{Soutput}
\begin{Sinput}
> quantile(X,c(0.25, 0.5, 0.75))
\end{Sinput}
\begin{Soutput}
25% 50% 75% 
  1   2   3 
\end{Soutput}
\begin{Sinput}
> quantile(X, 0.6)
\end{Sinput}
\begin{Soutput}
60% 
  2 
\end{Soutput}
\begin{Sinput}
> resumen <- summary(X); resumen
\end{Sinput}
\begin{Soutput}
   Min. 1st Qu.  Median    Mean 3rd Qu.    Max. 
    0.0     1.0     2.0     2.1     3.0     5.0 
\end{Soutput}
\begin{Sinput}
> fivenum(X)
\end{Sinput}
\begin{Soutput}
[1] 0 1 2 3 5
\end{Soutput}
\begin{Sinput}
> barplot(tfre[[2]], main="Gráfico de barras", xlab="X = Número Hijos\n", ylab="frecuencia",
+ col=c("yellow", "blue", "white", "orange", "cyan", "red"), sub="Agosto-2012")
> pie(tfre[[2]], main="Gráfico de pastel", xlab="Número Hijos \n", col=c("yellow", "blue",
+ "white", "orange", "cyan", "red"), sub="Agosto-2012")
> names(fab) = c("Cero", "Uno", "Dos", "Tres", "Cuatro", "Cinco")
> pie(fab, main="Gráfico de pastel", xlab="X = Número Hijos\n", col=c("yellow", "blue",
+ "white", "orange", "cyan", "red"), sub="Agosto-2012")
> boxplot(X, main="Gráfico de caja", ylab="Número de hijos\n")
> boxplot(X, main="Gráfico de caja", xlab=" Número de hijos\n", plot=TRUE, border="red",col="yellow", horizontal=TRUE)
> 
> 
> 
\end{Sinput}
\end{Schunk}



\textbf{Practica 8.}


Notas<-c(4.47,4.47,3.48,5.0,3.42,3.78,3.1,3.57,4.2,4.5,3.6,3.75,4.5,2.85,3.7,4.2,3.2,4.05,4.9,5.1,5.3,4.16,4.56,3.54,3.5,5.2,4.71,3.7,4.78,4.14,4.14,4.8,4.1,3.83,3.6,2.98,4.32,5.1,4.3,3.9,3.96,3.54,4.8,4.3,3.39,4.47,3.19,3.75,3.1,4.7,3.69,3.3,2.85,5.25,4.68,4.04,4.44,5.43,3.04,2.95);Notas
data.entry(Notas)
Notas
length(Notas)

write(Notas, "Notas.txt")

ls()
rm(list=ls(all=TRUE))
ls()

X <- scan("Notas.txt", what = double(0), na.strings = "NA", flush=FALSE)
ls()

n <- length(X); n
k <- 1+3.322*logb(60, 10); k
k <- round(k); k
rango <- max(X)-min(X); rango
a=rango/k; a
a <- round(a, 3); a

# Calcula el ancho o amplitud a de cada intervalo a=rango/k
rango <- max(X)-min(X); rango
a=rango/k; a
a <- round(a, 3); a


# Define los límites y puntos medios de cada uno de los k intervalos
limites <- seq(from=min(X)-0.01/2, to=max(X)+0.01/2, by=a); limites
options(digits=4)
ci <- cbind(1:k); ci
for(i in 2:length(limites)) ci[i-1, 1] <- (limites[i] + limites[i-1])/2
ci

# Encuentra las frecuencias absolutas fi para cada intervalo

options(digits=2)
fi <- cbind(table(cut(X, breaks = limites, labels=NULL, include.lowest=FALSE,
right=FALSE, dig.lab=4))); fi


# Encuentra las frecuencias relativas o proporciones fri

options(digits=4)
fri <- fi/n; fri

# Encuentra las frecuencias acumuladas ascendentes Fi
options(digits=2)
Fi <- cumsum(fi); Fi

# Encuentra las frecuencias relativas acumuladas Fri
options(digits=4)
Fri <- Fi/n; Fri

# Completa la tabla de frecuencias.
tablaFrec <- data.frame(ci=ci, fi=fi, fri=fri, Fi=Fi, Fri=Fri); tablaFrec

h <- hist(X, breaks=c(limites[1]-a, limites, limites[k+1]+a), freq = TRUE, probability = FALSE,
include.lowest = FALSE,right = TRUE, main = "Histograma de frecuencias",
col="lightyellow", lty=1, border="purple", xlab=" Notas de aspirantes", ylab="Frecuencia (fi)",
axes=TRUE, labels=FALSE)
text(h$mids, h$density, h$counts, adj=c(0.5, -0.5), col="red")
rug(jitter(X)) # adiciona marcas de los datos

# h es un objeto del tipo lista que contiene atributos del histograma
is.list(h); h

h <- hist(X, breaks=c(limites[1]-a, limites, limites[k+1]+a), freq = FALSE,
probability = TRUE, include.lowest = FALSE, right = TRUE,
main="Aproximación a una Normal\n", col="lightyellow",lty=1,border="purple",
xlab="Notas de aspirantes\n", ylab="Frecuencia relativa (fri)",
axes=TRUE, labels=FALSE)
text(h$mids, h$density, h$counts, adj=c(0.5, 0.2), col="red")
rug(jitter(X)) # adiciona marcas de los datos
curve(dnorm(x, mean=mean(X), sd=sd(X)), col = 2, lty = 2,lwd = 2, add = TRUE)


#Crea el polígono de frecuencias
h <- hist(X, breaks=c(limites[1]-a, limites, limites[k+1]+a), freq = TRUE,
probability=FALSE, include.lowest=FALSE,right=TRUE,
main = "Polígono de frecuencias",col="lightyellow", lty=1, border="purple", xlab="
Notas de aspirantes", ylab="Frecuencia (fi)", axes=TRUE, labels=FALSE)
text(h$mids, h$density, h$counts, adj=c(0.5, -0.5), col="red")
rug(jitter(X)) # adiciona marcas de los datos
vCi <- c(h$mids[1]-a, h$mids, h$mids[k+1]+a); vCi
vfi <- c(0, h$counts, 0); vfi
lines(vCi, vfi, col="blue", type="l")

#Crea la Ojiva ascendente o polígono de frecuencias acumuladas ascendentes
Fia <- c(0, Fi); Fia
plot(limites, Fia, type = "p", pch=1, col = "blue", main="Ojiva ascendente",
xlab="Notas de aspirantes", ylab="Frecuencia acumulada (Fi)")
text(limites, h$density, Fia, adj=c(0.5, -0.5), col="red")
lines(limites, Fia, col="black", type="l")

#Calcula los principales estadísticos descriptivos de la variable
# Calcula la moda, ya que el R no proporciona una función para eso.
options(digits=4)
for(i in 1:k) if (fi[i] == max(fi)) break()
if(i > 1) moda <- limites[i]+((fi[i]-fi[i-1])/((fi[i]-fi[i-1])+(fi[i]-fi[i+1]) ))*a
 else  moda <- limites[i]+(fi[i]/(fi[i]+(fi[i]-fi[i+1])))*a
moda

# Calcula los cuartiles: Q1, Q2, Q3
Q <- 1:3
for(v in 1:3) for(i in 1:k) if (Fi[i] > (v*25*n)/100)
{
Q[v] <- limites[i]+(((25*v*n/100)-Fi[i-1])/fi[i])*a
break
}
Q

#Calcula los principales estadísticos.
estadisticos <- rbind(media=sum(tabEstad$cifi)/n, moda=moda, Q1=Q[1], Q2=Q[2], Q3=Q[3],
rango=max(X)-min(X), varianza=sum(tabEstad$ciMedia2fi)/n,
Desviacion=sqrt(sum(tabEstad$ciMedia2fi)/n),
CoeficienteVariacion=sqrt(sum(tabEstad$ciMedia2fi)/n)/(sum(tabEstad$cifi)/n),
CAfisher=(sum(tabEstad$ciMedia3fi)/n)/sqrt(sum(tabEstad$ciMedia2fi)/n)^3,
CoeficienteCurtosis=((sum(tabEstad$ciMedia4fi)/n)/sqrt(sum(tabEstad$ciMedia2fi)/n)^4)-3)
estadisticos

# Gráfico de cajas
boxplot(X, main="Gráfico de caja", xlab="Notas", notch=FALSE,
data=parent.frame(), plot=TRUE, border="red", col="yellow",horizontal=TRUE)
#Observación: en la función boxplot(), sí plot es FALSE se produce un resumen de los valores (los cinco números).


windows()
boxplot(X, main="Gráfico de caja", xlab="X = Notas", notch=TRUE,
data=parent.frame(), plot=TRUE, border="red", col="yellow",horizontal=TRUE)

par(mfrow=c(1,2)) # Divide la ventana gráfica en dos partes (1 fila, 2 columnas)
mtext(side=3, line=0, cex=2, outer=T, "Titulo para Toda la Página")
hist(X); boxplot(X)

#Calcula los principales estad?sticos descriptivos de la variable
# Calcula la moda, ya que el R no proporciona una funci?n para eso.
options(digits=4)
for(i in 1:k) if (fi[i] == max(fi)) break()
if(i > 1) moda <- limites[i]+((fi[i]-fi[i-1])/((fi[i]-fi[i-1])+(fi[i]-fi[i+1]) ))*a
moda <- limites[i]+(fi[i]/(fi[i]+(fi[i]-fi[i+1])))*a
moda

#Varios gr?ficos en una misma ventana
par(mfrow=c(1,2)) # Divide la ventana gr?fica en dos partes (1 fila, 2 columnas)
mtext(side=3, line=0, cex=2, outer=T, "Titulo para Toda la P?gina")
hist(X); boxplot(X)




\textbf{Practica 9.}


library(foreign)
HojaCat <- read.table("HojaCat.txt", header=TRUE)
HojaCat


#Conecta la hoja de datos a la segunda ruta o lista de búsqueda.
attach(HojaCat, pos=2) # pos especifica la posición donde buscar la conexión
search()

#Crea una tabla de contigencia o de doble entrada
tablaCont <- table(HojaCat); tablaCont
length(HojaCat)


# Encuentra la suma de cada fila de la tabla de contingencia
# Distribución marginal de X=Estado civil
suma.filas <- apply(tablaCont, 1, sum); suma.filas
# El 1 indica que son totales por fila

# Gráficos de barras para tabla de contingencia.
# Barras apiladas
barplot(t(tablaCont), main="Gráfico de barras (Estado, Ocupación)", xlab="Estado civil", ylab="Ocupación", legend.text=TRUE)

barplot(t(tablaCont), main="Gráfico de barras (Estado, Ocupación)", xlab="Estado civil", ylab="Ocupación", beside=TRUE, legend.text=TRUE)

# Guardar las todas las opciones iniciales y modificar número de decimales
op <- options()
options(digits=3) # sólo imprime 3 lugares decimales
options('digits')


# Proporciones basadas en el total de la muestra, la suma de filas y columnas suman 1

propTotal <- prop.table(tablaCont); propTotal

barplot(t(propTotal), main="Gráfico de barras (Estado, Ocupación)", xlab="Estado civil\n",ylab="Ocupación", beside=TRUE, legend.text=TRUE)


# Proporciones basadas en el total por fila, cada fila suma 1.

propFila <- prop.table(tablaCont, 1); propFila
# Total por fila se indica en 1
barplot(t(propFila), main="Gráfico de barras (Estado, Ocupación)", xlab="Estado civil\n",
ylab="Ocupación", beside=TRUE, legend.text=TRUE)


propFila <- prop.table(tablaCont, 1); propFila
# Total por fila se indica en 1
barplot(t(propFila), main="Gráfico de barras (Estado, Ocupación)", xlab="Estado civil\n",
ylab="Ocupación", beside=TRUE, legend.text=TRUE)

#Realizar la prueba o contraste Chi-cuadrado de independencia
prueba <- chisq.test(tablaCont); prueba

# Frecuencias absolutas esperadas para la prueba Chi-cuadrada
prueba$expected # fij = fi./No. column







\textbf{Practica 10.}


\begin{Schunk}
\begin{Sinput}
> A <- c(100,96,92,96,92); A
\end{Sinput}
\begin{Soutput}
[1] 100  96  92  96  92
\end{Soutput}
\begin{Sinput}
> B <- c(76,80,75,84,82); B
\end{Sinput}
\begin{Soutput}
[1] 76 80 75 84 82
\end{Soutput}
\begin{Sinput}
> C <- c(108,100,96,98,100); C
\end{Sinput}
\begin{Soutput}
[1] 108 100  96  98 100
\end{Soutput}
\begin{Sinput}
> Baterias <- data.frame(procesoA=A, procesoB=B, procesoC=C); Baterias
\end{Sinput}
\begin{Soutput}
  procesoA procesoB procesoC
1      100       76      108
2       96       80      100
3       92       75       96
4       96       84       98
5       92       82      100
\end{Soutput}
\begin{Sinput}
> # Para editar los datos puede utilizar la funci?n fix()
> fix(Baterias)
> write.table(Baterias, file="Baterias.txt", append=FALSE, quote=TRUE, sep=" ", na="NA",
+ col.names=TRUE)
> ls(); rm(list=ls(all=TRUE)); ls()
\end{Sinput}
\begin{Soutput}
 [1] "A"        "B"        "Baterias" "C"        "cuasivar" "fab"     
 [7] "Fac"      "Far"      "fre"      "i"        "media"    "mediana" 
[13] "moda"     "resumen"  "s"        "tabla"    "tfre"     "X"       
\end{Soutput}
\begin{Soutput}
character(0)
\end{Soutput}
\begin{Sinput}
> Baterias <- read.table("Baterias.txt", header=TRUE); Baterias
\end{Sinput}
\begin{Soutput}
  procesoA procesoB procesoC
1      100       76      108
2       96       80      100
3       92       75       96
4       96       84       98
5       92       82      100
\end{Soutput}
\begin{Sinput}
> attach(Baterias, pos=2)
> search()
\end{Sinput}
\begin{Soutput}
 [1] ".GlobalEnv"        "Baterias"          "package:RcmdrMisc"
 [4] "package:sandwich"  "package:car"       "package:splines"  
 [7] "package:stats"     "package:graphics"  "package:grDevices"
[10] "package:utils"     "package:datasets"  "package:methods"  
[13] "Autoloads"         "package:relimp"    "package:base"     
\end{Soutput}
\begin{Sinput}
> stripchart(Baterias, main="Gr?fico de puntos para los tres procesos", method = "stack", vertical =
+ FALSE, col="blue", pch=1, xlab="Duraci?n (semanas)", ylab="Proceso")
> #Muestra un resumen estad?stico para los tres procesos.
> summary(Baterias)
\end{Sinput}
\begin{Soutput}
    procesoA      procesoB     procesoC  
 Min.   : 92   Min.   :75   Min.   : 96  
 1st Qu.: 92   1st Qu.:76   1st Qu.: 98  
 Median : 96   Median :80   Median :100  
 Mean   : 95   Mean   :79   Mean   :100  
 3rd Qu.: 96   3rd Qu.:82   3rd Qu.:100  
 Max.   :100   Max.   :84   Max.   :108  
\end{Soutput}
\begin{Sinput}
> # Horizontal
> boxplot(Baterias, width=NULL, varwidth=TRUE, names, add= FALSE, horizontal = TRUE,
+ main="Gr?fico de caja por proceso", border=par("fg"), col=c("yellow", "cyan", "red"), xlab =
+ "Duraci?n (semanas)", ylab="Proceso")
> # Vertical
> boxplot(Baterias, width=NULL, varwidth=TRUE, names, add= FALSE, horizontal = FALSE,
+ main="Gr?fico de caja por proceso", border=par("fg"), col=c("yellow", "cyan", "red"), xlab =
+ "Duraci?n (semanas)", ylab="Proceso")
> #Presenta la matriz de covarianzas muestral.
> options(digits=3) # s?lo imprime 3 lugares decimales
> S <- var(Baterias); S
\end{Sinput}
\begin{Soutput}
         procesoA procesoB procesoC
procesoA     11.2     -1.6     12.4
procesoB     -1.6     14.8     -4.7
procesoC     12.4     -4.7     20.8
\end{Soutput}
\begin{Sinput}
> Baterias <- stack(Baterias); Baterias
\end{Sinput}
\begin{Soutput}
   values      ind
1     100 procesoA
2      96 procesoA
3      92 procesoA
4      96 procesoA
5      92 procesoA
6      76 procesoB
7      80 procesoB
8      75 procesoB
9      84 procesoB
10     82 procesoB
11    108 procesoC
12    100 procesoC
13     96 procesoC
14     98 procesoC
15    100 procesoC
\end{Soutput}
\begin{Sinput}
> names(Baterias) # Muestra los encabezados de los vectores
\end{Sinput}
\begin{Soutput}
[1] "values" "ind"   
\end{Soutput}
\begin{Sinput}
> #Desconecta la hoja de datos de la segunda ruta o lista de b?squeda.
> detach(Baterias, pos=2); search()
\end{Sinput}
\begin{Soutput}
 [1] ".GlobalEnv"        "package:RcmdrMisc" "package:sandwich" 
 [4] "package:car"       "package:splines"   "package:stats"    
 [7] "package:graphics"  "package:grDevices" "package:utils"    
[10] "package:datasets"  "package:methods"   "Autoloads"        
[13] "package:relimp"    "package:base"     
\end{Soutput}
\begin{Sinput}
> #An?lisis de una variable bidimensional
> 
> Fuma = c("Si","No","No","Si","No","Si","Si","Si","No","Si"); Fuma
\end{Sinput}
\begin{Soutput}
 [1] "Si" "No" "No" "Si" "No" "Si" "Si" "Si" "No" "Si"
\end{Soutput}
\begin{Sinput}
> Cantidad = c(1,2,2,3,3,1,2,1,3,2); Cantidad
\end{Sinput}
\begin{Soutput}
 [1] 1 2 2 3 3 1 2 1 3 2
\end{Soutput}
\begin{Sinput}
> Estudia <- data.frame(Fuma=Fuma, Cantidad=Cantidad); Estudia
\end{Sinput}
\begin{Soutput}
   Fuma Cantidad
1    Si        1
2    No        2
3    No        2
4    Si        3
5    No        3
6    Si        1
7    Si        2
8    Si        1
9    No        3
10   Si        2
\end{Soutput}
\begin{Sinput}
> fix(Estudia)
> write.table(Estudia, file="Estudia.txt", append=FALSE, quote=TRUE, sep=" ", na="NA",
+ col.names=TRUE)
> write.table
\end{Sinput}
\begin{Soutput}
function (x, file = "", append = FALSE, quote = TRUE, sep = " ", 
    eol = "\n", na = "NA", dec = ".", row.names = TRUE, col.names = TRUE, 
    qmethod = c("escape", "double"), fileEncoding = "") 
{
    qmethod <- match.arg(qmethod)
    if (is.logical(quote) && (length(quote) != 1L || is.na(quote))) 
        stop("'quote' must be 'TRUE', 'FALSE' or numeric")
    quoteC <- if (is.logical(quote)) 
        quote
    else TRUE
    qset <- is.logical(quote) && quote
    if (!is.data.frame(x) && !is.matrix(x)) 
        x <- data.frame(x)
    makeRownames <- isTRUE(row.names)
    makeColnames <- is.logical(col.names) && !identical(FALSE, 
        col.names)
    if (is.matrix(x)) {
        p <- ncol(x)
        d <- dimnames(x)
        if (is.null(d)) 
            d <- list(NULL, NULL)
        if (is.null(d[[1L]]) && makeRownames) 
            d[[1L]] <- seq_len(nrow(x))
        if (is.null(d[[2L]]) && makeColnames && p > 0L) 
            d[[2L]] <- paste0("V", 1L:p)
        if (qset) 
            quote <- if (is.character(x)) 
                seq_len(p)
            else numeric()
    }
    else {
        if (qset) 
            quote <- if (length(x)) 
                which(unlist(lapply(x, function(x) is.character(x) || 
                  is.factor(x))))
            else numeric()
        if (any(sapply(x, function(z) length(dim(z)) == 2 && 
            dim(z)[2L] > 1))) {
            c1 <- names(x)
            x <- as.matrix(x, rownames.force = makeRownames)
            d <- dimnames(x)
            if (qset) {
                ord <- match(c1, d[[2L]], 0L)
                quote <- ord[quote]
                quote <- quote[quote > 0L]
            }
        }
        else d <- list(if (makeRownames) row.names(x), if (makeColnames) names(x))
        p <- ncol(x)
    }
    nocols <- p == 0L
    if (is.logical(quote)) 
        quote <- NULL
    else if (is.numeric(quote)) {
        if (any(quote < 1L | quote > p)) 
            stop("invalid numbers in 'quote'")
    }
    else stop("invalid 'quote' specification")
    rn <- FALSE
    rnames <- NULL
    if (is.logical(row.names)) {
        if (row.names) {
            rnames <- as.character(d[[1L]])
            rn <- TRUE
        }
    }
    else {
        rnames <- as.character(row.names)
        rn <- TRUE
        if (length(rnames) != nrow(x)) 
            stop("invalid 'row.names' specification")
    }
    if (!is.null(quote) && rn) 
        quote <- c(0, quote)
    if (is.logical(col.names)) {
        if (!rn && is.na(col.names)) 
            stop("'col.names = NA' makes no sense when 'row.names = FALSE'")
        col.names <- if (is.na(col.names) && rn) 
            c("", d[[2L]])
        else if (col.names) 
            d[[2L]]
        else NULL
    }
    else {
        col.names <- as.character(col.names)
        if (length(col.names) != p) 
            stop("invalid 'col.names' specification")
    }
    if (file == "") 
        file <- stdout()
    else if (is.character(file)) {
        file <- if (nzchar(fileEncoding)) 
            file(file, ifelse(append, "a", "w"), encoding = fileEncoding)
        else file(file, ifelse(append, "a", "w"))
        on.exit(close(file))
    }
    else if (!isOpen(file, "w")) {
        open(file, "w")
        on.exit(close(file))
    }
    if (!inherits(file, "connection")) 
        stop("'file' must be a character string or connection")
    qstring <- switch(qmethod, escape = "\\\\\"", double = "\"\"")
    if (!is.null(col.names)) {
        if (append) 
            warning("appending column names to file")
        if (quoteC) 
            col.names <- paste("\"", gsub("\"", qstring, col.names), 
                "\"", sep = "")
        writeLines(paste(col.names, collapse = sep), file, sep = eol)
    }
    if (nrow(x) == 0L) 
        return(invisible())
    if (nocols && !rn) 
        return(cat(rep.int(eol, NROW(x)), file = file, sep = ""))
    if (is.matrix(x) && !is.atomic(x)) 
        mode(x) <- "character"
    if (is.data.frame(x)) {
        x[] <- lapply(x, function(z) {
            if (is.object(z) && !is.factor(z)) 
                as.character(z)
            else z
        })
    }
    invisible(.External2(C_writetable, x, file, nrow(x), p, rnames, 
        sep, eol, na, dec, as.integer(quote), qmethod != "double"))
}
<bytecode: 0x000000000af8a920>
<environment: namespace:utils>
\end{Soutput}
\begin{Sinput}
> ls()
\end{Sinput}
\begin{Soutput}
[1] "Baterias" "Cantidad" "Estudia"  "Fuma"     "S"       
\end{Soutput}
\begin{Sinput}
> rm(list=ls(all=TRUE))
> ls()
\end{Sinput}
\begin{Soutput}
character(0)
\end{Soutput}
\begin{Sinput}
> Estudia <- read.table("Estudia.txt", header=TRUE)
> Estudia
\end{Sinput}
\begin{Soutput}
   Fuma Cantidad
1    Si        1
2    No        2
3    No        2
4    Si        3
5    No        3
6    Si        1
7    Si        2
8    Si        1
9    No        3
10   Si        2
\end{Soutput}
\begin{Sinput}
> tablaCont <- table(Estudia)
> tablaCont
\end{Sinput}
\begin{Soutput}
    Cantidad
Fuma 1 2 3
  No 0 2 2
  Si 3 2 1
\end{Soutput}
\begin{Sinput}
> options(digits=3) # s?lo imprime 3 lugares decimales
> propTotal <- prop.table(tablaCont); propTotal
\end{Sinput}
\begin{Soutput}
    Cantidad
Fuma   1   2   3
  No 0.0 0.2 0.2
  Si 0.3 0.2 0.1
\end{Soutput}
\begin{Sinput}
> propFila <- prop.table(tablaCont, 1)
> propFila
\end{Sinput}
\begin{Soutput}
    Cantidad
Fuma     1     2     3
  No 0.000 0.500 0.500
  Si 0.500 0.333 0.167
\end{Soutput}
\begin{Sinput}
> propCol <- prop.table(tablaCont, 2)
> propCol
\end{Sinput}
\begin{Soutput}
    Cantidad
Fuma     1     2     3
  No 0.000 0.500 0.667
  Si 1.000 0.500 0.333
\end{Soutput}
\begin{Sinput}
> barplot(table(Estudia$Cantidad, Estudia$Fuma), beside = FALSE, horizontal=FALSE, main="Gr?fico de barras (Fuma, Cantidad de horas de estudio)", legend.text =T, xlab="Fuma", ylab="Cantidad de horas-estudio")
> barplot(table(Estudia$Fuma, Estudia$Cantidad), beside = FALSE, horizontal=FALSE,main="Gr?fico
+ de barras (Cantidad de horas de estudio,Fuma)", legend.text =T, xlab="Cantidad de horas-estudio",
+ ylab="Fuma")
> Fuma=factor(Estudia$Fuma); Fuma
\end{Sinput}
\begin{Soutput}
 [1] Si No No Si No Si Si Si No Si
Levels: No Si
\end{Soutput}
\begin{Sinput}
> barplot(table(Estudia$Cantidad, Estudia$Fuma), main="Gr?fico de barras (Fuma, Cantidad de horas
+ de estudio)", xlab="Fuma", ylab="Cantidad de horas-estudio", beside=TRUE, legend.text=T)
> barplot(table(Estudia$Cantidad, Estudia$Fuma), main="Gr?fico de barras (Fuma, Cantidad de horas
+ de estudio)", xlab="Fuma", ylab="Cantidad de horas-estudio", beside=TRUE, legend.text=c("menor
+ que 5", "5-10", "mayor que 10"))
> 
> 
> 
\end{Sinput}
\end{Schunk}



\textbf{Practica 11.}

\begin{Schunk}
\begin{Sinput}
> usuarios <- c(10, 15, 20, 20, 25, 30, 30); usuarios
\end{Sinput}
\begin{Soutput}
[1] 10 15 20 20 25 30 30
\end{Soutput}
\begin{Sinput}
> tiempo = c(1.0, 1.2, 2.0, 2.1, 2.2, 2.0, 1.9); tiempo
\end{Sinput}
\begin{Soutput}
[1] 1.0 1.2 2.0 2.1 2.2 2.0 1.9
\end{Soutput}
\begin{Sinput}
> Sistema <- data.frame(Usuarios=usuarios, Tiempo=tiempo);Sistema
\end{Sinput}
\begin{Soutput}
  Usuarios Tiempo
1       10    1.0
2       15    1.2
3       20    2.0
4       20    2.1
5       25    2.2
6       30    2.0
7       30    1.9
\end{Soutput}
\begin{Sinput}
> fix(Sistema)
> write.table(Sistema, file="Sistema.txt", append=FALSE, quote=TRUE, sep=" ", na="NA",
+ col.names = TRUE)
> ls(); rm(list=ls(all=TRUE)); ls()
\end{Sinput}
\begin{Soutput}
[1] "Estudia"   "Fuma"      "propCol"   "propFila"  "propTotal" "Sistema"  
[7] "tablaCont" "tiempo"    "usuarios" 
\end{Soutput}
\begin{Soutput}
character(0)
\end{Soutput}
\begin{Sinput}
> #Recupera la hoja de datos.
> Sistema <- read.table("Sistema.txt", header=TRUE); Sistema
\end{Sinput}
\begin{Soutput}
  Usuarios Tiempo
1       10    1.0
2       15    1.2
3       20    2.0
4       20    2.1
5       25    2.2
6       30    2.0
7       30    1.9
\end{Soutput}
\begin{Sinput}
> #Conecta la hoja de datos a la segunda ruta o lista de búsqueda.
> attach(Sistema, pos=2); search()
\end{Sinput}
\begin{Soutput}
 [1] ".GlobalEnv"        "Sistema"           "package:RcmdrMisc"
 [4] "package:sandwich"  "package:car"       "package:splines"  
 [7] "package:stats"     "package:graphics"  "package:grDevices"
[10] "package:utils"     "package:datasets"  "package:methods"  
[13] "Autoloads"         "package:relimp"    "package:base"     
\end{Soutput}
\begin{Sinput}
> #Muestra un resumen de principales estadísticos de las variables.
> summary(Sistema)
\end{Sinput}
\begin{Soutput}
    Usuarios        Tiempo    
 Min.   :10.0   Min.   :1.00  
 1st Qu.:17.5   1st Qu.:1.55  
 Median :20.0   Median :2.00  
 Mean   :21.4   Mean   :1.77  
 3rd Qu.:27.5   3rd Qu.:2.05  
 Max.   :30.0   Max.   :2.20  
\end{Soutput}
\begin{Sinput}
> cov(Sistema) # Matriz de covarianzas
\end{Sinput}
\begin{Soutput}
         Usuarios Tiempo
Usuarios    55.95  2.714
Tiempo       2.71  0.222
\end{Soutput}
\begin{Sinput}
> cor(Sistema, use = "all.obs", method="pearson") # Matriz de correlaciones
\end{Sinput}
\begin{Soutput}
         Usuarios Tiempo
Usuarios    1.000  0.769
Tiempo      0.769  1.000
\end{Soutput}
\begin{Sinput}
> #Elabora un gráfico de dispersión para analizar alguna relación entre las variables.
> plot(Usuarios, Tiempo, xlim= c(5, 35), ylim= c(0.0, 2.5), type = "p", pch=1, col = "blue", main =
+ "Gráfico de dispersión (Usuarios, Tiempo)", xlab="Número de usuarios", ylab="Tiempo de
+ ejecución")
> #Sin cerrar la ventana del gráfico anterior, ejecuta la siguiente instrucción
> identify(Usuarios, Tiempo, n=1) # n=1 indica que solamente será un punto seleccionado
\end{Sinput}
\begin{Soutput}
integer(0)
\end{Soutput}
\begin{Sinput}
> reg.Y.X <- lm(Tiempo ~ -1 + Usuarios, Sistema, na.action=NULL, method="qr", model=TRUE)
> #-1 indica que no se toma en cuenta la constante en el modelo.
> summary(reg.Y.X)
\end{Sinput}
\begin{Soutput}
Call:
lm(formula = Tiempo ~ -1 + Usuarios, data = Sistema, na.action = NULL, 
    method = "qr", model = TRUE)

Residuals:
   Min     1Q Median     3Q    Max 
-0.483 -0.187  0.206  0.313  0.511 

Coefficients:
         Estimate Std. Error t value Pr(>|t|)    
Usuarios   0.0794     0.0065    12.2  1.8e-05 ***
---
Signif. codes:  0 '***' 0.001 '**' 0.01 '*' 0.05 '.' 0.1 ' ' 1

Residual standard error: 0.387 on 6 degrees of freedom
Multiple R-squared:  0.961,	Adjusted R-squared:  0.955 
F-statistic:  150 on 1 and 6 DF,  p-value: 1.82e-05
\end{Soutput}
\begin{Sinput}
> lines(Usuarios, 0.079437*Usuarios)
> reg.anova <- anova(reg.Y.X); reg.anova
\end{Sinput}
\begin{Soutput}
Analysis of Variance Table

Response: Tiempo
          Df Sum Sq Mean Sq F value  Pr(>F)    
Usuarios   1   22.4   22.40     150 1.8e-05 ***
Residuals  6    0.9    0.15                    
---
Signif. codes:  0 '***' 0.001 '**' 0.01 '*' 0.05 '.' 0.1 ' ' 1
\end{Soutput}
\begin{Sinput}
> 
> 
> 
\end{Sinput}
\end{Schunk}



\textbf{Practica 13.}


\begin{Schunk}
\begin{Sinput}
> moneda <- c("C", "+"); moneda
\end{Sinput}
\begin{Soutput}
[1] "C" "+"
\end{Soutput}
\begin{Sinput}
> n <- 10; n
\end{Sinput}
\begin{Soutput}
[1] 10
\end{Soutput}
\begin{Sinput}
> espacio <- 1:54;espacio
\end{Sinput}
\begin{Soutput}
 [1]  1  2  3  4  5  6  7  8  9 10 11 12 13 14 15 16 17 18 19 20 21 22 23 24 25
[26] 26 27 28 29 30 31 32 33 34 35 36 37 38 39 40 41 42 43 44 45 46 47 48 49 50
[51] 51 52 53 54
\end{Soutput}
\begin{Sinput}
> # se define el tamaño de la muestra
> n <- 6; n
\end{Sinput}
\begin{Soutput}
[1] 6
\end{Soutput}
\begin{Sinput}
> muestra <- sample(espacio, n); muestra
\end{Sinput}
\begin{Soutput}
[1] 31  2 34 44  4 49
\end{Soutput}
\begin{Sinput}
> # genera el espacio muestral del lanzamiendo de los dos dados
> espacio = as.vector(outer(1:6, 1:6, paste)); espacio
\end{Sinput}
\begin{Soutput}
 [1] "1 1" "2 1" "3 1" "4 1" "5 1" "6 1" "1 2" "2 2" "3 2" "4 2" "5 2" "6 2"
[13] "1 3" "2 3" "3 3" "4 3" "5 3" "6 3" "1 4" "2 4" "3 4" "4 4" "5 4" "6 4"
[25] "1 5" "2 5" "3 5" "4 5" "5 5" "6 5" "1 6" "2 6" "3 6" "4 6" "5 6" "6 6"
\end{Soutput}
\begin{Sinput}
> # se define el tamaño de la muestra
> n <- 4; n
\end{Sinput}
\begin{Soutput}
[1] 4
\end{Soutput}
\begin{Sinput}
> # finalmente se selecciona la muestra
> muestra <- sample(espacio, n, replace=TRUE); muestra
\end{Sinput}
\begin{Soutput}
[1] "2 5" "5 5" "2 1" "3 2"
\end{Soutput}
\begin{Sinput}
> #genera el espacio muestral de las 52 cartas
> naipe = paste(rep(c("A", 2:10, "J", "Q", "K"), 4), c("OROS","COPAS", "BASTOS",
+ "ESPADAS"));naipe
\end{Sinput}
\begin{Soutput}
 [1] "A OROS"     "2 COPAS"    "3 BASTOS"   "4 ESPADAS"  "5 OROS"    
 [6] "6 COPAS"    "7 BASTOS"   "8 ESPADAS"  "9 OROS"     "10 COPAS"  
[11] "J BASTOS"   "Q ESPADAS"  "K OROS"     "A COPAS"    "2 BASTOS"  
[16] "3 ESPADAS"  "4 OROS"     "5 COPAS"    "6 BASTOS"   "7 ESPADAS" 
[21] "8 OROS"     "9 COPAS"    "10 BASTOS"  "J ESPADAS"  "Q OROS"    
[26] "K COPAS"    "A BASTOS"   "2 ESPADAS"  "3 OROS"     "4 COPAS"   
[31] "5 BASTOS"   "6 ESPADAS"  "7 OROS"     "8 COPAS"    "9 BASTOS"  
[36] "10 ESPADAS" "J OROS"     "Q COPAS"    "K BASTOS"   "A ESPADAS" 
[41] "2 OROS"     "3 COPAS"    "4 BASTOS"   "5 ESPADAS"  "6 OROS"    
[46] "7 COPAS"    "8 BASTOS"   "9 ESPADAS"  "10 OROS"    "J COPAS"   
[51] "Q BASTOS"   "K ESPADAS" 
\end{Soutput}
\begin{Sinput}
> # se define el tamaño de la muestra
> n <- 5; n
\end{Sinput}
\begin{Soutput}
[1] 5
\end{Soutput}
\begin{Sinput}
> # se obtiene la muestra sin reemplazo (aunque no se especifique con replace=FALSE)
> cartas <- sample(naipe, n) ; cartas
\end{Sinput}
\begin{Soutput}
[1] "4 ESPADAS"  "2 BASTOS"   "10 ESPADAS" "8 BASTOS"   "5 COPAS"   
\end{Soutput}
\begin{Sinput}
> espacio <- function(num)
+ {
+ numDiv7 <- numeric(0)
+ ind <- 0
+ for(i in 1:length(num))
+ if ((num[i] %% 7)==0)
+ {
+ ind <- ind+1
+ numDiv7[ind]=num[i]
+ }
+ return(numDiv7)
+ }
> numeros <- 1:500
> # generando el espacio muestral
> s <- espacio(numeros); s
\end{Sinput}
\begin{Soutput}
 [1]   7  14  21  28  35  42  49  56  63  70  77  84  91  98 105 112 119 126 133
[20] 140 147 154 161 168 175 182 189 196 203 210 217 224 231 238 245 252 259 266
[39] 273 280 287 294 301 308 315 322 329 336 343 350 357 364 371 378 385 392 399
[58] 406 413 420 427 434 441 448 455 462 469 476 483 490 497
\end{Soutput}
\begin{Sinput}
> # seleccionando la muestra
> muestra <- sample(s, 12, replace=TRUE); muestra
\end{Sinput}
\begin{Soutput}
 [1] 392 476 112 322 224 112 273 308 273 476 406 441
\end{Soutput}
\begin{Sinput}
> 
> 
> 
> 
> 
> 
> 
\end{Sinput}
\end{Schunk}



\textbf{Practica 14.}

\begin{Schunk}
\begin{Sinput}
> dbinom(4,8,0.5)
\end{Sinput}
\begin{Soutput}
[1] 0.273
\end{Soutput}
\begin{Sinput}
> x <- 2; n=8; p=1/2
> pbinom(x, size = n, prob = p, lower.tail=TRUE)
\end{Sinput}
\begin{Soutput}
[1] 0.145
\end{Soutput}
\begin{Sinput}
> x <- 4; n=8; p=1/2
> #primera forma
> F <- 1 - pbinom(x, n, p, lower.tail=TRUE); F
\end{Sinput}
\begin{Soutput}
[1] 0.363
\end{Soutput}
\begin{Sinput}
> #segunda forma
> pbinom(4, size=8, prob=0.5, lower.tail=FALSE)
\end{Sinput}
\begin{Soutput}
[1] 0.363
\end{Soutput}
\begin{Sinput}
> x <- 3; mu <- 6
> ppois(x, lambda = mu, lower.tail=TRUE)
\end{Sinput}
\begin{Soutput}
[1] 0.151
\end{Soutput}
\begin{Sinput}
> #primera forma
> 
> sum(dpois(c(6,7,8),lambda = 6))
\end{Sinput}
\begin{Soutput}
[1] 0.402
\end{Soutput}
\begin{Sinput}
> # segunda forma
> F8 <- ppois(8, lambda = 6, lower.tail=TRUE)
> F5 <- ppois(5,lambda = 6, lower.tail=TRUE)
> F8 - F5
\end{Sinput}
\begin{Soutput}
[1] 0.402
\end{Soutput}
\begin{Sinput}
> n <- 30
> #genera 30 valores de una distribución de Poisson con <U+03BB> = 6
> x <- rpois(n, lambda=mu)
> #calcula las probabilidades para cada valor generado
> 
> 
> y <- dpois(x, lambda=mu)
> #genera el gráfico de distribución
> plot(x, y, xlab="x", ylab="Función de probalidad", main="Distribución de Poisson: lambda = 6",
+ type="h")
> #une los puntos a las líneas
> points(x, y, pch=21)
> x <- 0:2 
> m = 11
> n <- 4; k=2
> # x define el número de globos con premio
> 
> 
> 
> 
\end{Sinput}
\end{Schunk}


\textbf{Practica 15 y 16.}
\begin{Schunk}
\begin{Sinput}
> #ejemplo 1
> x <-55
> a=0
> b<-90
> punif(x,min=a,max=b,lower.tail=TRUE)
\end{Sinput}
\begin{Soutput}
[1] 0.611
\end{Soutput}
\begin{Sinput}
> ## [1] 0.6111111
> F55=punif(55,min=a,max=b,lower.tail=TRUE)
> F15=punif(15,min=a,max=b,lower.tail=TRUE)
> F55-F15
\end{Sinput}
\begin{Soutput}
[1] 0.444
\end{Soutput}
\begin{Sinput}
> ## [1] 0.4444444
> (1-F55)*( F55-F15)
\end{Sinput}
\begin{Soutput}
[1] 0.173
\end{Soutput}
\begin{Sinput}
> ## [1] 0.1728395
> #ejemplo 2
> p<-c(0.80)
> media=5;
> d.t=1
> qnorm(p, mean=media,sd=d.t,lower.tail=TRUE)
\end{Sinput}
\begin{Soutput}
[1] 5.84
\end{Soutput}
\begin{Sinput}
> ## [1] 5.841621
> p<-c(0.80)
> g.l<-10
> qt(p,df=g.l,lower.tail=TRUE)
\end{Sinput}
\begin{Soutput}
[1] 0.879
\end{Soutput}
\begin{Sinput}
> ## [1] 0.8790578
> n<-16
> x<-4.5
> mu=5
> sigma=1
> d.t=sigma/sqrt(n)
> pnorm(x,mean=mu,sd=d.t,lower.tail=FALSE)
\end{Sinput}
\begin{Soutput}
[1] 0.977
\end{Soutput}
\begin{Sinput}
> ## [1] 0.9772499
> #ejemplo 3
> x<-5
> teta=7
> pexp(x,rate=1/teta,lower.tail=FALSE)
\end{Sinput}
\begin{Soutput}
[1] 0.49
\end{Soutput}
\begin{Sinput}
> ## [1] 0.4895417
> x<-3
> teta=7
> pexp(x,rate=1/teta,lower.tail=TRUE)
\end{Sinput}
\begin{Soutput}
[1] 0.349
\end{Soutput}
\begin{Sinput}
> ## [1] 0.3485609
> pexp(4,rate=1/teta,lower.tail=FALSE)
\end{Sinput}
\begin{Soutput}
[1] 0.565
\end{Soutput}
\begin{Sinput}
> ## [1] 0.5647181
> p<-0.9
> teta<-7
> qexp(p,rate=1/teta,lower.tail=TRUE)
\end{Sinput}
\begin{Soutput}
[1] 16.1
\end{Soutput}
\begin{Sinput}
> ## [1] 16.1181
> qexp(0.5,rate=1/teta,lower.tail=TRUE)
\end{Sinput}
\begin{Soutput}
[1] 4.85
\end{Soutput}
\begin{Sinput}
> ## [1] 4.85203
> qexp(0.68,rate=1/teta,lower.tail=TRUE)
\end{Sinput}
\begin{Soutput}
[1] 7.98
\end{Soutput}
\begin{Sinput}
> ## [1] 7.97604
> qexp(0.32,rate=1/teta,lower.tail=FALSE)
\end{Sinput}
\begin{Soutput}
[1] 7.98
\end{Soutput}
\begin{Sinput}
> ## [1] 7.97604
> 
> #3. gneracion de muestras aleatorias de las distribuciones
> #ejemplo 1
> min<--2
> max<-4
> x=runif(100,min,max)
> x
\end{Sinput}
\begin{Soutput}
  [1] -0.0661 -0.8637  2.8351  3.8520  0.9152  1.2254 -0.6538 -0.3348 -1.0595
 [10]  1.3927  1.7962 -1.8701  0.0289 -1.5401  1.9786  2.4962  1.6495 -1.4762
 [19]  2.4177  2.9380 -0.0157 -0.9551  2.5848  1.2963  3.1225  2.1285 -1.2859
 [28]  0.2262  0.3336 -1.2697  0.1288  1.8772  0.8349 -0.0833  1.9781  0.3206
 [37]  3.1574  0.3545  3.8122 -0.6939 -0.8646 -1.3100 -1.5346  3.1197  2.3764
 [46] -1.9411 -1.6917 -0.6441  2.3946  2.1240  2.9227  1.3271 -0.3180  2.9157
 [55]  1.9165  1.7584  2.8197  2.8564 -1.2610 -1.0879  2.3474  0.1023  3.6113
 [64] -1.3778  3.3620 -1.0535  3.5934  2.6475  2.8704  0.2943  1.6337  2.2372
 [73]  0.2296  1.9543  2.0687 -1.3273  2.2443  1.2530  2.6443  1.6825  2.6824
 [82] -1.1341  0.7599  3.2404 -1.9529  2.8798  3.3304 -0.2669 -0.0771 -1.4145
 [91]  2.6021  0.9822  1.9817  0.1260 -1.1657 -0.2112  2.6197 -1.5545 -0.9013
[100]  0.2056
\end{Soutput}
\begin{Sinput}
> hist(x,main="X ~Uniforme(min=-2,max=4",xlab="X",ylab="densidad de probabilidad",
+ probability=TRUE,col="cyan")
> curve(dunif(x,min,max),col="blue",add=TRUE)
> #ejemplo 2
> x.norm<-rnorm(n=200,mean=10,sd=2)
> x.norm
\end{Sinput}
\begin{Soutput}
  [1] 10.39 10.06 12.27 13.49  8.08  7.60  9.59  9.41  9.46  9.98 11.77  5.51
 [13] 10.76  9.77  7.85 10.74 12.27  9.80 11.67 13.46  8.75  7.51  9.32 12.61
 [25] 13.11 15.42 11.37  9.85  8.95  9.07 13.57  9.98 11.52  8.39 12.76  8.56
 [37] 10.33 12.52  9.06  8.64 12.14 10.95  8.34 11.60  9.06 10.93  9.47  7.57
 [49]  7.26  9.92  8.48  6.25 10.58 10.09 11.36 10.66  9.78  9.73  6.54  7.53
 [61] 10.06 10.13 10.37  8.96  9.73  9.65  9.32  9.05 10.07  9.38  8.59  7.62
 [73]  8.84  7.45 12.60  5.42 11.08 10.70  9.25  7.87 10.37  9.49  9.27  9.26
 [85]  8.66 10.87 10.81 12.27  5.97  8.86 10.57  8.09 10.93 13.02 12.23 11.27
 [97] 10.58  9.13  8.62  6.14  9.81  8.93  7.69  8.44 11.42  8.09  8.27 10.20
[109] 13.30  8.32  8.53  9.09  7.67  9.32 10.40 10.37 12.38  8.70 11.12 10.93
[121] 10.42 10.23  9.04 13.22  9.91  9.00  9.23  9.59 11.27 10.36  8.33  9.42
[133] 12.71 10.75 11.95 11.04 11.22 10.03  8.67  8.99 10.25 13.94 17.02 10.78
[145]  7.85 11.88  9.00 11.07  6.26 14.44 10.71 10.82 13.51 10.96  9.70 12.76
[157]  8.83 10.17 10.62  8.97 11.05  9.62  7.40  9.34 10.10 10.60 10.93 13.70
[169] 10.98 10.63 10.09  7.98  9.41  8.88 11.13  9.08 11.65 10.10  5.36 11.51
[181] 10.44  5.56  8.58  9.80  9.28 10.95  8.79  7.43 10.01  6.57  8.83 11.61
[193] 10.69 11.22 11.19 16.01  8.93 10.61 11.62  9.45
\end{Soutput}
\begin{Sinput}
> hist(x.norm,breaks="Sturges",freq=TRUE,probability=FALSE,include.lowest=TRUE,right
+ =TRUE,density=NULL,angle=45,col="steelblue1",border = NULL, main = "Histograma de
+ datos observados",axes=TRUE,plot=TRUE,labels=FALSE)
> plot(ecdf(x.norm),main="Funcion de distribucion acumulada teorica")
\end{Sinput}
\end{Schunk}

\textbf{Practica 17.}



este es un ejemplo
\begin{Schunk}
\begin{Sinput}
> simulIntProp <- function(m=5, n=1, p, nivel.conf=0.95)
+ {
+ X <- rbinom(m, n, p)
+ # Matriz con 1000 valores aleatorios binomial(n,p), 50 muestras cada una de tama?o 20
+ pe <<- X/n
+ # Calcula la proporci?n estimada en cada una de las muestras.
+ SE <<- sqrt(pe*(1-pe)/n)
+ # Calcula la desviaci?n est?ndar estimada en cada una de las muestras.
+ alfa <- 1-nivel.conf
+ z <<- qnorm(1-alfa/2)
+ Intervalo <<- cbind(pe - z*SE, pe + z*SE)
+ # genera los extremos del intervalo de confianza
+ nInter <<- 0
+ # un contador para conocer en cu?ntos intervalos se encuentra la verdadera proporci?n.
+ for(i in 1:m)
+ if ((p >= Intervalo[i, 1]) && (p <= Intervalo[i, 2]))
+ nInter <<- nInter + 1
+ # funci?n que cuenta cu?ntos intervalos contienen el verdadero valor del par?metro.
+ return(nInter)
+ }
> n=20; m= 50; p=0.5; nivel.conf=0.95
> simulIntProp(m, n, p, nivel.conf)
\end{Sinput}
\begin{Soutput}
[1] 47
\end{Soutput}
\begin{Sinput}
> Intervalo # para visualizar cada uno de los intervalos generados
\end{Sinput}
\begin{Soutput}
        [,1]  [,2]
 [1,] 0.3320 0.768
 [2,] 0.0247 0.375
 [3,] 0.2809 0.719
 [4,] 0.5602 0.940
 [5,] 0.2809 0.719
 [6,] 0.3320 0.768
 [7,] 0.2320 0.668
 [8,] 0.1853 0.615
 [9,] 0.3320 0.768
[10,] 0.3853 0.815
[11,] 0.3853 0.815
[12,] 0.1853 0.615
[13,] 0.1853 0.615
[14,] 0.2809 0.719
[15,] 0.1410 0.559
[16,] 0.2809 0.719
[17,] 0.3320 0.768
[18,] 0.1853 0.615
[19,] 0.3853 0.815
[20,] 0.3320 0.768
[21,] 0.2809 0.719
[22,] 0.2320 0.668
[23,] 0.1410 0.559
[24,] 0.5602 0.940
[25,] 0.3320 0.768
[26,] 0.1853 0.615
[27,] 0.2809 0.719
[28,] 0.3853 0.815
[29,] 0.2320 0.668
[30,] 0.1410 0.559
[31,] 0.2809 0.719
[32,] 0.2320 0.668
[33,] 0.4410 0.859
[34,] 0.3320 0.768
[35,] 0.0992 0.501
[36,] 0.2320 0.668
[37,] 0.4410 0.859
[38,] 0.4992 0.901
[39,] 0.3853 0.815
[40,] 0.2809 0.719
[41,] 0.3320 0.768
[42,] 0.3853 0.815
[43,] 0.3853 0.815
[44,] 0.2320 0.668
[45,] 0.1410 0.559
[46,] 0.1853 0.615
[47,] 0.4410 0.859
[48,] 0.1853 0.615
[49,] 0.3320 0.768
[50,] 0.3853 0.815
\end{Soutput}
\begin{Sinput}
> nInter
\end{Sinput}
\begin{Soutput}
[1] 47
\end{Soutput}
\begin{Sinput}
> # para visualizar en cu?ntos de estos intervalos se encuentra la verdadera proporci?n.
> #Gr?fico que muestra los intervalos de confianza de 95% que contienen y no contienen el verdadero valor del par?metro p.
> matplot(rbind(pe - z*SE, pe + z*SE), rbind(1:m, 1:m), type="l", lty=1)
> abline(v=p)
\end{Sinput}
\end{Schunk}



\textbf{Practica 18.}


\begin{Schunk}
\begin{Sinput}
> intervaloProp <- function(x, n, nivel.conf=0.95)
+ {
+ pe <- x/n
+ alfa <- 1-nivel.conf
+ z <- qnorm(1-alfa/2)
+ SE <- sqrt(pe*(1-pe)/n)
+ print(rbind(pe, alfa, z, SE))
+ LInf <- pe-z*SE
+ LSup <- pe+z*SE
+ print(" ")
+ print(paste("Intervalo para p es: [", round(LInf, 2), ",", round(LSup, 2), "]"))
+ }
> x=360; n=1200; nivel.conf=0.95
> intervaloProp(x, n, nivel.conf)
\end{Sinput}
\begin{Soutput}
       [,1]
pe   0.3000
alfa 0.0500
z    1.9600
SE   0.0132
[1] " "
[1] "Intervalo para p es: [ 0.27 , 0.33 ]"
\end{Soutput}
\end{Schunk}



\textbf{Practica 21.}
\documentclass{article}

\begin{document}

PRUEBAS DE NORMALIDAD DE UNA MUESTRA


Se digitan los datos del grupo de control 

\begin{Schunk}
\begin{Sinput}
> IMC_Control <- c(23.6, 22.7, 21.2, 21.7, 20.7, 22.0, 21.8, 24.2, 20.1, 21.3, 
+                  20.5, 21.1, 21.4, 22.2, 22.6, 20.4, 23.3, 24.8)
> par(mfrow=c(1,2)) 
\end{Sinput}
\end{Schunk}


Se genera el histograma de la variables de interés

\begin{Schunk}
\begin{Sinput}
> hist(IMC_Control,main="A",xlab="IMC (kg/m2)",ylab="Frecuencia")
\end{Sinput}
\end{Schunk}


Se genera el diagrama de caja de la variable de interés y se muestra en la misma ventana

\begin{Schunk}
\begin{Sinput}
> boxplot(IMC_Control,main="B", lab="IMC (kg/m2)",ylim=c(20,25)) 
\end{Sinput}
\end{Schunk}



Los commandos para contrastar normalidad son los siguientes

\begin{Schunk}
\begin{Sinput}
> sw <- shapiro.test(IMC_Control)
> sw 
\end{Sinput}
\begin{Soutput}
	Shapiro-Wilk normality test

data:  IMC_Control
W = 1, p-value = 0.5
\end{Soutput}
\begin{Sinput}
> ks <- ks.test(IMC_Control,"pnorm",mean=mean(IMC_Control),sd=sd(IMC_Control))
> ks
\end{Sinput}
\begin{Soutput}
	One-sample Kolmogorov-Smirnov test

data:  IMC_Control
D = 0.1, p-value = 1
alternative hypothesis: two-sided
\end{Soutput}
\end{Schunk}

Luego se digitan los datos para pacientes y se ejecutan las mismas instrucciones
\begin{Schunk}
\begin{Sinput}
> IMC_Pacientes <- c(25.6, 22.7, 25.9, 24.3, 25.2, 29.6, 21.3, 25.5, 27.4, 
+                    22.3, 24.4, 23.7, 20.6, 22.8) 
> par(mfrow=c(1,2))
> hist(IMC_Pacientes,main="A",xlab="IMC (kg/m2)",ylab="Frecuencia")
> boxplot(IMC_Pacientes,main="B", lab="IMC (kg/m2)",ylim=c(20,30))
> sw <- shapiro.test(IMC_Pacientes)
> sw 
\end{Sinput}
\begin{Soutput}
	Shapiro-Wilk normality test

data:  IMC_Pacientes
W = 1, p-value = 0.9
\end{Soutput}
\begin{Sinput}
> ks <- ks.test(IMC_Pacientes,"pnorm",mean=mean(IMC_Pacientes),sd=sd(IMC_Pacientes))
> ks
\end{Sinput}
\begin{Soutput}
	One-sample Kolmogorov-Smirnov test

data:  IMC_Pacientes
D = 0.1, p-value = 1
alternative hypothesis: two-sided
\end{Soutput}
\end{Schunk}



\end{document}
\textbf{Practica 22.}
\documentclass{article}

\begin{document}

Construyendo una función en R para realizar la prueba de hipótesis.

\begin{Schunk}
\begin{Sinput}
> Prueba.prop <- function(x, n, po, H1="Distinto", alfa=0.05)
+ {
+   op <- options();
+   options(digits=2)
+   pe=x/n 
+   SE <- sqrt((po * (1-po))/n) 
+   Zo <- (pe-po)/SE 
+   
+   if (H1 == "Menor" || H1 == "Mayor")
+   {
+     Z <- qnorm(alfa, mean=0, sd=1, lower.tail = FALSE, log.p = FALSE)
+     
+     valores <- rbind(Prop_Estimada=pe, Prop_Hipotetica=po, Z_critico=Z,Estadistico= Zo)
+   }
+   else
+   {
+     Z <- qnorm(alfa/2, mean=0, sd=1, lower.tail = FALSE, log.p = FALSE)
+     
+     valores <- rbind(Prop_Estimada=pe, Prop_Hipotetica =po, Z_critico_menor=-Z,
+                      Z_critico_mayor =Z, Zo)
+   } 
+   if (H1 == "Menor")
+   {
+     if (Zo < -Z) decision <- paste("Como Estadistico <", round(-Z,3), 
+                                    ", entonces rechazamos Ho")
+     else decision <- paste("Como Estadistico>=", round(-Z,3), 
+                            ", entonces aceptamos Ho")
+   }
+   if (H1 == "Mayor")
+   {
+     if (Zo > Z) decision <- paste("Como Estadistico >", round(Z,3), 
+                                   ", entonces rechazamos Ho")
+     else decision <- paste("Como Estadistico <=", round(Z,3), 
+                            ", entonces aceptamos Ho")
+   }
+   if (H1 == "Distinto")
+   {
+     if (Zo < -Z) decision <- paste("Como Estadistico <", round(-Z,3), 
+                                    ", entonces rechazamos Ho")
+     if (Zo > Z) decision <- paste("Como Estadistico >", round(Z,3), 
+                                   ", entonces rechazamos Ho")
+     else decision <- paste("Como Estadistico pertenece a [", round(-Z,3), 
+ ",", round(Z,3), "], entonces aceptamos Ho") 
+   } 
+   print(valores)
+   print(decision)
+   options(op) 
+ } 
> Prueba.prop(23, 100, 0.15, H1="Menor", alfa=0.05)
\end{Sinput}
\begin{Soutput}
                [,1]
Prop_Estimada   0.23
Prop_Hipotetica 0.15
Z_critico       1.64
Estadistico     2.24
[1] "Como Estadistico>= -1.645 , entonces aceptamos Ho"
\end{Soutput}
\begin{Sinput}
> Prueba.prop(23, 100, 0.15, H1="Mayor", alfa=0.05)
\end{Sinput}
\begin{Soutput}
                [,1]
Prop_Estimada   0.23
Prop_Hipotetica 0.15
Z_critico       1.64
Estadistico     2.24
[1] "Como Estadistico > 1.645 , entonces rechazamos Ho"
\end{Soutput}
\begin{Sinput}
> Prueba.prop(23, 100, 0.15, H1="Distinto", alfa=0.05)
\end{Sinput}
\begin{Soutput}
                 [,1]
Prop_Estimada    0.23
Prop_Hipotetica  0.15
Z_critico_menor -1.96
Z_critico_mayor  1.96
Zo               2.24
[1] "Como Estadistico > 1.96 , entonces rechazamos Ho"
\end{Soutput}
\begin{Sinput}
> prop.test(x=23, n=100, p=0.15, alternative="less", conf.level=0.95)
\end{Sinput}
\begin{Soutput}
	1-sample proportions test with continuity correction

data:  23 out of 100, null probability 0.15
X-squared = 4, df = 1, p-value = 1
alternative hypothesis: true p is less than 0.15
95 percent confidence interval:
 0.000 0.311
sample estimates:
   p 
0.23 
\end{Soutput}
\begin{Sinput}
> prop.test(x=23, n=100, p=0.15, alternative="greater", conf.level=0.95)
\end{Sinput}
\begin{Soutput}
	1-sample proportions test with continuity correction

data:  23 out of 100, null probability 0.15
X-squared = 4, df = 1, p-value = 0.02
alternative hypothesis: true p is greater than 0.15
95 percent confidence interval:
 0.164 1.000
sample estimates:
   p 
0.23 
\end{Soutput}
\begin{Sinput}
> prop.test(x=23, n=100, p=0.15, alternative="two.sided", conf.level=0.95) 
\end{Sinput}
\begin{Soutput}
	1-sample proportions test with continuity correction

data:  23 out of 100, null probability 0.15
X-squared = 4, df = 1, p-value = 0.04
alternative hypothesis: true p is not equal to 0.15
95 percent confidence interval:
 0.154 0.327
sample estimates:
   p 
0.23 
\end{Soutput}
\end{Schunk}



PRUEBA DE HIPÓTESIS SOBRE UNA MEDIA, VARIANZA CONOCIDA.

\begin{Schunk}
\begin{Sinput}
> X <- c(9.0, 3.41, 6.13, 1.99, 6.92, 3.12, 7.86, 2.01, 5.98, 
+        4.15, 6.87, 1.97, 4.01, 3.56, 8.04, 3.24, 5.05, 7.37)
> Prueba.param <- function(x, des, VEC, H1="Distinto", alfa=0.05)
+ {
+   op <- options();
+   options(digits=2)
+   miu<- mean(VEC)
+   RC<- 1.645
+   L<- length(VEC)
+   Zo<-(miu-x)/(((des^2)/L)^(0.5))
+   Z<- 1.645
+  
+   if (H1 == "Menor")
+   {
+     if (Zo < -Z) decision <- paste("Como Estadistico <", round(-Z,3), 
+                                    ", entonces rechazamos Ho")
+     else decision <- paste("Como Estadistico>=", round(-Z,3), 
+                            ", entonces aceptamos Ho")
+   }
+   if (H1 == "Mayor")
+   {
+     if (Zo > Z) decision <- paste("Como Estadistico >", round(Z,3), 
+                                   ", entonces rechazamos Ho")
+     else decision <- paste("Como Estadistico <=", round(Z,3), 
+                            ", entonces aceptamos Ho")
+   }
+   if (H1 == "Distinto")
+   {
+     if (Zo < -Z) decision <- paste("Como Estadistico <", round(-Z,3), 
+                                    ", entonces rechazamos Ho")
+     if (Zo > Z) decision <- paste("Como Estadistico >", round(Z,3), 
+                                   ", entonces rechazamos Ho")
+     else decision <- paste("Como Estadistico pertenece a [", round(-Z,3), 
+ ",", round(Z,3), "], entonces aceptamos Ho") 
+   } 
+   
+   print(decision)
+   options(op) 
+   } 
> Prueba.param(4, 2.45, X, H1="Mayor", alfa=0.05)
\end{Sinput}
\begin{Soutput}
[1] "Como Estadistico > 1.645 , entonces rechazamos Ho"
\end{Soutput}
\begin{Sinput}
> t.test(X,mu=4,alternative="greater") 
\end{Sinput}
\begin{Soutput}
	One Sample t-test

data:  X
t = 2, df = 20, p-value = 0.04
alternative hypothesis: true mean is greater than 4
95 percent confidence interval:
 4.1 Inf
sample estimates:
mean of x 
     5.04 
\end{Soutput}
\end{Schunk}


\end{document}
\textbf{Practica 23.}
\documentclass{article}

\begin{document}


\begin{Schunk}
\begin{Sinput}
> # GUIA 23
> #PRUEBAS SOBRE DOS MUESTRAS INDEPENDIENTES
> IMC_Control <- c(23.6, 22.7, 21.2, 21.7, 20.7, 22.0, 21.8, 24.2, 20.1,
+                  21.3, 20.5, 21.1, 21.4, 22.2, 22.6,
+                  20.4, 23.3, 24.8)
> IMC_Pacientes <- c(25.6, 22.7, 25.9, 24.3, 25.2, 29.6, 21.3, 25.5, 27.4, 22.3,
+                    24.4, 23.7, 20.6, 22.8)
> t.test(IMC_Control, IMC_Pacientes, var.equal=TRUE, mu=0)
\end{Sinput}
\begin{Soutput}
	Two Sample t-test

data:  IMC_Control and IMC_Pacientes
t = -4, df = 30, p-value = 0.001
alternative hypothesis: true difference in means is not equal to 0
95 percent confidence interval:
 -3.77 -1.03
sample estimates:
mean of x mean of y 
     22.0      24.4 
\end{Soutput}
\begin{Sinput}
> #Se concluye entonces que existe diferencia significativa en el IMC para ambos grupos
> #de pacientes,#pues el p valor de la prueba resulta ser muy peque<f1>o.
> 
> #PRUEBAS SOBRE DOS MUESTRAS PAREADAS
> PAS.antes <- c(160,155,180,140,150,130,190,192,170,165)
> PAS.despues <- c(139,135,175,120,145,140,170,180,149,146)
> shapiro.test(PAS.antes)
\end{Sinput}
\begin{Soutput}
	Shapiro-Wilk normality test

data:  PAS.antes
W = 1, p-value = 0.9
\end{Soutput}
\begin{Sinput}
> shapiro.test(PAS.despues)
\end{Sinput}
\begin{Soutput}
	Shapiro-Wilk normality test

data:  PAS.despues
W = 0.9, p-value = 0.4
\end{Soutput}
\begin{Sinput}
> ks.test(PAS.antes,"pnorm",mean=mean(PAS.antes),sd=sd(PAS.antes))
\end{Sinput}
\begin{Soutput}
	One-sample Kolmogorov-Smirnov test

data:  PAS.antes
D = 0.1, p-value = 1
alternative hypothesis: two-sided
\end{Soutput}
\begin{Sinput}
> ks.test(PAS.despues,"pnorm",mean=mean(PAS.despues),sd=sd(PAS.despues))
\end{Sinput}
\begin{Soutput}
	One-sample Kolmogorov-Smirnov test

data:  PAS.despues
D = 0.2, p-value = 0.6
alternative hypothesis: two-sided
\end{Soutput}
\begin{Sinput}
> t.test(PAS.antes, PAS.despues, paired=TRUE, mu=0)
\end{Sinput}
\begin{Soutput}
	Paired t-test

data:  PAS.antes and PAS.despues
t = 4, df = 9, p-value = 0.003
alternative hypothesis: true difference in means is not equal to 0
95 percent confidence interval:
  5.88 20.72
sample estimates:
mean of the differences 
                   13.3 
\end{Soutput}
\begin{Sinput}
> #El valor del estad<ed>stico t es 4.0552, con gl = 9, P = 0.0029. Con estos resultados
> #se rechaza 0 H y por lo tanto se concluye que la PAS antes y despu<e9>s del 
> #tratamiento es distinta, es decir, el tratamiento ha sido efectivo.
> 
> #PRUEBA DE HIP<d3>TESIS ACERCA DE LA VARIANZA DE DOS POBLACIONES
> Agente_A <- c(12, 11, 18, 16, 13)
> Agente_B <- c(14, 18, 18, 17, 16)
> var.test(Agente_A, Agente_B)
\end{Sinput}
\begin{Soutput}
	F test to compare two variances

data:  Agente_A and Agente_B
F = 3, num df = 4, denom df = 4, p-value = 0.3
alternative hypothesis: true ratio of variances is not equal to 1
95 percent confidence interval:
  0.316 29.157
sample estimates:
ratio of variances 
              3.04 
\end{Soutput}
\begin{Sinput}
> #Como el p valor es alto se concluye que las varianzas pueden considerarse iguales.
> 
> 
> #EJERCICIO
> #contraste de la igualdad de varianzas
> Tabla_A <- c(2098,2082,2246,2340,2714,2777,2625,2388,2766,3112,3030,3375,3038,
+              3017,3136,3204,3174,3220,3464,3870,3689,3783,3457,4151,4230,3707,
+              415,4315,4790,4464,4499,4819,4739,4912,4494,5698,6349,6630,7585,8183)
> Tabla_B<- c(1209,1115,1151,1208,1170,1198,1390,1480,1359,1337,1415,1530,1453,
+             1324,1477,1501,1661,1562,1764,1796,1976,1802,2000,1923,2097,2110,
+             2214,2069,2324,2309,2353,2091,2187,2399,2630,2722,2998,3392,3379,3627)
> var.test(Tabla_A, Tabla_B)
\end{Sinput}
\begin{Soutput}
	F test to compare two variances

data:  Tabla_A and Tabla_B
F = 5, num df = 40, denom df = 40, p-value = 6e-07
alternative hypothesis: true ratio of variances is not equal to 1
95 percent confidence interval:
  2.88 10.31
sample estimates:
ratio of variances 
              5.45 
\end{Soutput}
\begin{Sinput}
> #Como el p valor es bajo se concluye que las varianzas pueden considerarse distintas.
> 
> #contraste de igualdad de medias.
> t.test(Tabla_A, Tabla_B, var.equal=TRUE, mu=0)
\end{Sinput}
\begin{Soutput}
	Two Sample t-test

data:  Tabla_A and Tabla_B
t = 7, df = 80, p-value = 3e-10
alternative hypothesis: true difference in means is not equal to 0
95 percent confidence interval:
 1375 2419
sample estimates:
mean of x mean of y 
     3815      1918 
\end{Soutput}
\begin{Sinput}
> #Se concluye que existe diferencia significativa en 
> #la densidad espectral para ambos grupos de pacientes,
> #pues el p valor de la prueba resulta ser muy peque<f1>o.
\end{Sinput}
\end{Schunk}



\end{document}
\textbf{Practica 24.}
\documentclass{article}

\begin{document}


\begin{Schunk}
\begin{Sinput}
> #GUIA 24
> #AN<c1>LISIS DE VARIANZA
> 
> #Ejemplo 1
> 
> notas <- c(20,18,18,23,22,17,15,13,21,15,20,13,12,16,17,21,15,13,12,15,18,20,
+            18,17,10,24,16)
> programas <- gl(n=3, k=9, labels=c("P1", "P2", "P3"))
> datos <- data.frame(notas = notas, programas = programas);datos
\end{Sinput}
\begin{Soutput}
   notas programas
1     20        P1
2     18        P1
3     18        P1
4     23        P1
5     22        P1
6     17        P1
7     15        P1
8     13        P1
9     21        P1
10    15        P2
11    20        P2
12    13        P2
13    12        P2
14    16        P2
15    17        P2
16    21        P2
17    15        P2
18    13        P2
19    12        P3
20    15        P3
21    18        P3
22    20        P3
23    18        P3
24    17        P3
25    10        P3
26    24        P3
27    16        P3
\end{Soutput}
\begin{Sinput}
> mod1 <- aov(notas ~ programas, data = datos)
> plot(mod1)
> 
> #Ejemplo 2
> #(No esta la base de datos)
> 
\end{Sinput}
\end{Schunk}



\end{document}
\textbf{Practica 25 y 26.}

\begin{Schunk}
\begin{Sinput}
> bloques<-gl(n=4,k=1,length = 20)
> bloques # Vector de bloque del experimento
\end{Sinput}
\begin{Soutput}
 [1] 1 2 3 4 1 2 3 4 1 2 3 4 1 2 3 4 1 2 3 4
Levels: 1 2 3 4
\end{Soutput}
\begin{Sinput}
> tratamientos<-gl(n=5,k=4)
> tratamientos # Vector de tratamientos de los novillos
\end{Sinput}
\begin{Soutput}
 [1] 1 1 1 1 2 2 2 2 3 3 3 3 4 4 4 4 5 5 5 5
Levels: 1 2 3 4 5
\end{Soutput}
\begin{Sinput}
> peso<-c(0.9,1.4,1.4,2.3,3.6,3.2,4.5,4.1,0.5,0.9,0.5,0.9,3.6,3.6,3.2,3.6,1.8,1.8,0.9,1.4)
> peso # Se han registrado los pesos de Novillos
\end{Sinput}
\begin{Soutput}
 [1] 0.9 1.4 1.4 2.3 3.6 3.2 4.5 4.1 0.5 0.9 0.5 0.9 3.6 3.6 3.2 3.6 1.8 1.8 0.9
[20] 1.4
\end{Soutput}
\begin{Sinput}
> datos2<-data.frame(bloques=bloques,tratamientos=tratamientos,peso=peso)
> datos2 # Se ha registrado en una hoja de datos los resultados del experimento
\end{Sinput}
\begin{Soutput}
   bloques tratamientos peso
1        1            1  0.9
2        2            1  1.4
3        3            1  1.4
4        4            1  2.3
5        1            2  3.6
6        2            2  3.2
7        3            2  4.5
8        4            2  4.1
9        1            3  0.5
10       2            3  0.9
11       3            3  0.5
12       4            3  0.9
13       1            4  3.6
14       2            4  3.6
15       3            4  3.2
16       4            4  3.6
17       1            5  1.8
18       2            5  1.8
19       3            5  0.9
20       4            5  1.4
\end{Soutput}
\begin{Sinput}
> mod2<-aov(peso ~ tratamientos + bloques,data = datos2)  # Se aplica el análisis de varianza
> summary(mod2) # Muestra la tabla ANOVA del tratamiento
\end{Sinput}
\begin{Soutput}
             Df Sum Sq Mean Sq F value  Pr(>F)    
tratamientos  4  30.71    7.68   39.11 8.6e-07 ***
bloques       3   0.46    0.15    0.78    0.53    
Residuals    12   2.36    0.20                    
---
Signif. codes:  0 '***' 0.001 '**' 0.01 '*' 0.05 '.' 0.1 ' ' 1
\end{Soutput}
\begin{Sinput}
> FactorA<-gl(n=4,k=8,length = 32)
> FactorA # Definiendo el vector que contiene el Factor A
\end{Sinput}
\begin{Soutput}
 [1] 1 1 1 1 1 1 1 1 2 2 2 2 2 2 2 2 3 3 3 3 3 3 3 3 4 4 4 4 4 4 4 4
Levels: 1 2 3 4
\end{Soutput}
\begin{Sinput}
> FactorB<-gl(n=4,k=2,length = 32)
> FactorB # Definiendo el vector que contiene los tratamientos de los novillos
\end{Sinput}
\begin{Soutput}
 [1] 1 1 2 2 3 3 4 4 1 1 2 2 3 3 4 4 1 1 2 2 3 3 4 4 1 1 2 2 3 3 4 4
Levels: 1 2 3 4
\end{Soutput}
\begin{Sinput}
> Porcentaje<-c(1.8,2.1,2.0,2.1,4.6,5.0,7.5,7.9,2.2,2.4,4.2,4.0,5.4,5.6,9.8,9.2,2.8,3.2,4.4,4.8,8.7,8.4,13.2,13.0,3.2,3.6,3.3,3.5,5.7,5.8,10.9,11.1)
> Porcentaje # Definiendo los pesos de los novillos
\end{Sinput}
\begin{Soutput}
 [1]  1.8  2.1  2.0  2.1  4.6  5.0  7.5  7.9  2.2  2.4  4.2  4.0  5.4  5.6  9.8
[16]  9.2  2.8  3.2  4.4  4.8  8.7  8.4 13.2 13.0  3.2  3.6  3.3  3.5  5.7  5.8
[31] 10.9 11.1
\end{Soutput}
\begin{Sinput}
> datos3<-data.frame(FactorA=FactorA,FactorB=FactorB,Porcentaje=Porcentaje)
> datos3 # Registrando en una hoja los datos del resultado del experimento
\end{Sinput}
\begin{Soutput}
   FactorA FactorB Porcentaje
1        1       1        1.8
2        1       1        2.1
3        1       2        2.0
4        1       2        2.1
5        1       3        4.6
6        1       3        5.0
7        1       4        7.5
8        1       4        7.9
9        2       1        2.2
10       2       1        2.4
11       2       2        4.2
12       2       2        4.0
13       2       3        5.4
14       2       3        5.6
15       2       4        9.8
16       2       4        9.2
17       3       1        2.8
18       3       1        3.2
19       3       2        4.4
20       3       2        4.8
21       3       3        8.7
22       3       3        8.4
23       3       4       13.2
24       3       4       13.0
25       4       1        3.2
26       4       1        3.6
27       4       2        3.3
28       4       2        3.5
29       4       3        5.7
30       4       3        5.8
31       4       4       10.9
32       4       4       11.1
\end{Soutput}
\begin{Sinput}
> mod3<-aov(Porcentaje~FactorA*FactorB,data = datos3)
> summary(mod3) # Se muestra la tabla ANOVA del experimento
\end{Sinput}
\begin{Soutput}
                Df Sum Sq Mean Sq F value  Pr(>F)    
FactorA          3   41.9    14.0   279.2 5.0e-14 ***
FactorB          3  283.9    94.6  1892.9 < 2e-16 ***
FactorA:FactorB  9   15.9     1.8    35.2 7.1e-09 ***
Residuals       16    0.8     0.1                    
---
Signif. codes:  0 '***' 0.001 '**' 0.01 '*' 0.05 '.' 0.1 ' ' 1
\end{Soutput}
\begin{Sinput}
> 
\end{Sinput}
\end{Schunk}



\end{document}
